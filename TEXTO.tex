\chapter*{Oração aos moços}
\hedramarkboth{oração aos moços}{rui barbosa}

\textsc{Senhores:}

Não quis Deus que os meus cinquenta anos de consagração ao direito
viessem receber no templo do seu ensino\footnote{ Rui refere"-se à
Academia de Direito de São Paulo, criada em 1827 e primeira
instituição a integrar a Universidade de São Paulo no momento de sua
criação, em 1934. O primeiro reitor da \textsc{usp}, Reynaldo Porchat,
era docente da Faculdade de Direito e nela sediou"-se a Reitoria
naqueles primeiros tempos. Vale lembrar ainda que \textit{Oração}, aqui, 
está no sentido de discurso, geralmente pronunciado em ocasião solene. [As notas numéricas são 
do organizador e as indicadas por letras, do próprio Rui Barbosa. N. do E.]} 
em S.~Paulo o selo de uma grande bênção, associando"-se hoje com a 
vossa admissão ao nosso sacerdócio, na solenidade imponente dos votos em que o ides esposar.

Em verdade vos digo, jovens amigos meus, que o coincidir desta
existência declinante com essas carreiras nascentes agora, o seu
coincidir num ponto de intersecção tão magnificamente celebrado, era
mais do que eu merecia; e, negando"-me a divina bondade um momento de
tamanha ventura, não me negou senão o a que eu não devia ter tido a
inconsciência de aspirar.

Mas, recusando"-me o privilégio de um dia tão grande, ainda me
consentiu o encanto de vos falar, de conversar convosco, presente entre
vós em espírito; o que é, também, estar presente em verdade.

Assim que não me ides ouvir de longe, como a quem se sente
arredado por centenas de quilômetros, mas de ao pé, de em meio a vós,
como a quem está debaixo do mesmo teto, e à beira do mesmo lar, em
colóquio de irmãos, ou junto dos mesmos altares, sob os mesmos
campanários, elevando ao Criador as mesmas orações, e professando o
mesmo credo.

Direis que isto de me achar assistindo, assim, entre os de quem me
vejo separado por distância tão vasta, seria dar"-se, ou supor que se
está dando, no meio de nós, um verdadeiro milagre?

Será. Milagre do maior dos taumaturgos. Milagre de quem respira
entre milagres. Milagre de um santo, que cada qual tem no sacrário do
seu peito. Milagre do coração, que os sabe chover sobre a criatura
humana, como o firmamento chove nos campos mais áridos e tristes a
orvalhada das noites, que se esvai, com os sonhos de antemanhã, ao cair
das primeiras frechas de oiro\footnote{ Nas formas em que aparecem a
alternância entre os ditongos \textit{oi}/\textit{ou}, Rui
costumeiramente opta pela forma \textit{oi}, hoje, menos comum entre
brasileiros.} do disco solar.

Embora o realismo dos adágios teime no contrário, tolerem"-me o
arrojo de afrontar uma vez a sabedoria dos provérbios. Eu me abalanço a
lhes dizer e redizer de não. Não é certo, como corre mundo, ou, pelo
menos, muitas e muitíssimas vezes, não é verdade, como se espalha fama,
que ``longe da vista, longe do coração''.

O gênio dos anexins, aí, vai longe de andar certo. Esse prolóquio
tem mais malícia que ciência, mais epigrama que justiça, mais engenho
que filosofia. Vezes sem conto,\footnote{ \textit{sem conto}:  variação
de “sem conta”, que não se pode contar; inúmeros; em grande
quantidade.} quando se está mais fora da vista dos olhos, então (e por
isso mesmo) é que mais à vista do coração estamos; não só bem à sua
vista, senão bem dentro nele.

Não, filhos meus (deixai"-me experimentar, uma vez que seja,
convosco, este suavíssimo nome); não: o coração não é tão frívolo, tão
exterior, tão carnal, quanto se cuida. Há, nele, mais que um assombro
fisiológico: um prodígio moral. E o órgão da fé, o órgão da esperança,
o órgão do ideal. Vê, por isso, com os olhos d'alma, o
que não veem os do corpo. Vê ao longe, vê em ausência, vê no invisível,
e até no infinito vê. Onde para o cérebro de ver, outorgou"-lhe o Senhor
que ainda veja; e não se sabe até onde. Até onde chegam as vibrações do
sentimento, até onde se perdem os surtos da poesia, até onde se somem
os voos da crença: até Deus mesmo, inviso como os panoramas íntimos do
coração, mas presente ao céu e à terra, a todos nós presente,\footnote{ Apesar 
da edição \textsc{ol} (1921) manter a palavra “presentes” no plural,
optamos por empregá"-la no singular, como também fazem as edições \textsc{crb}
(1999), \textsc{adr} (1932), haja vista a sintaxe do trecho ser “(Deus) a todos
nós presente”.} enquanto nos palpite, incorrupto, no seio, o
músculo da vida e da nobreza e da bondade humana.

Quando ele já não estende o raio visual pelo horizonte do
invisível, quando sua visão tem por limite a do nervo óptico, é que o
coração, já esclerótico, ou degenerescente, e saturado nos resíduos de
uma vida gasta no mal, apenas oscila mecanicamente no interior do
arcaboiço, como pêndula de relógio abandonado, que agita, com as
derradeiras pancadas, os vermes e a poeira da caixa. Dele se retirou a
centelha divina. Até ontem lhe banhava ela de luz todo esse espaço, que
nos distancia do incomensurável desconhecido, e lançava entre este e
nós uma ponte de astros. Agora apagados esses luzeiros, que o inundavam
de radiosa claridade, lá se foram, com o extinto cintilar das estrelas,
as entreabertas do dia eterno, deixando"-nos, tão"-somente, entre o
longínquo mistério daquele termo e o aniquilamento da nossa miséria
desamparada, as trevas de outro éter, como esse que se diz encher de
escuridão o vago mistério do espaço.

Entre vós, porém, moços, que me estais escutando, ainda brilha em
toda a sua rutilância o clarão da lâmpada sagrada, ainda arde em toda a
sua energia o centro de calor, a que se aquece a essência
d'alma. Vosso coração, pois, ainda estará
incontaminado; e Deus assim o preserve.

Metei a mão no seio, e aí o sentireis com a sua segunda vista.
Desta, sobre tudo, é que ele nutre sua vida agitada e criadora. Pois
não sabemos que, com os antepassados, vive ele da memória, do luto e da
saudade? E tudo é viver no pretérito. Não sentimos como, com os nossos
conviventes, se alimenta ele na comunhão dos sentimentos e índoles, das
ideias e aspirações? E tudo é viver num mundo, em que estamos sempre
fora deste, pelo amor, pela abnegação, pelo sacrifício, pela caridade.
Não nos será claro que, com os nossos descendentes e sobreviventes, com
os nossos sucessores e pósteros, vive ele de fé, esperança e sonho?
Ora, tudo é viver, previvendo, é existir, preexistindo, é ver,
prevendo. E, assim, está o coração, cada ano, cada dia, cada hora,
sempre alimentado em contemplar o que não vê, por ter em dote dos céus
a preexcelência de ver, ouvir e palpar o que os olhos não divisam, os
ouvidos não escutam, e o tato não sente.

Para o coração, pois, não há passado, nem futuro, nem ausência.
Ausência, pretérito e porvir, tudo lhe é atualidade, tudo presença. Mas
presença animada e vivente, palpitante e criadora, neste regaço
interior, onde os mortos renascem, prenascem os vindoiros, e os
distanciados se ajuntam, ao influxo de um talismã, pelo qual, nesse
mágico microcosmo de maravilhas, encerrado na breve arca de um peito
humano, cabe, em evocações de cada instante, a humanidade toda e a
mesma eternidade.

A maior de quantas distâncias logre a imaginação conceber, é a da
morte; e nem esta separa entre si os que a terrível afastadora de
homens arrebatou aos braços uns dos outros. Quantas vezes não
entrevemos, nesse fundo obscuro e remotíssimo, uma imagem cara? quantas
vezes não a vemos assomar nos longes da saudade, sorridente, ou
melancólica, alvoroçada, ou inquieta, severa, ou carinhosa,
trazendo"-nos o bálsamo, ou o conselho, a promessa, ou o desengano, a
recompensa, ou o castigo, o aviso da fatalidade, ou os presságios de
bom agoiro? Quantas nos não vem conversar, afável e tranquila, ou
pressurosa e sobressaltada, com o afago nas mãos, a doçura na boca, a
meiguice no semblante, o pensamento na fronte, límpida, ou carregada, e
lhe saímos do contato, ora seguros e robustecidos, ora
transidos de cuidado e
pesadume, ora cheios de novas inspirações, e cismando, para a vida,
novos rumos? Quantas outras, não somos nós os que vamos chamar esses
leais companheiros de além"-mundo, e com eles renovar a
prática\footnote{ \textit{prática}: palestra, conferência, fala,
conversação.} interrompida, ou instar com eles por um alvitre, em
vão buscado, uma palavra, um movimento do rosto, um gesto, uma réstia
de luz, um traço do que por lá se sabe, e aqui se ignora?

Se não há, pois, abismo entre duas épocas, nem mesmo a voragem
final desta à outra vida, que não transponha a mútua atração de duas
almas, não pode haver, na mesquinha superfície do globo terrestre,
espaços, que não vença, com os instantâneos de presteza das vibrações
luminosas, esse fluido incomparável, por onde se realiza, na esfera das
comunicações morais, a maravilha da fotografia à distância\footnote{ O
uso gramatical baseado nos clássicos da língua é de que o sintagma \textit{a
distância}, quando a distância de que se fala não é especificada, se
grafe sem crase: ``Viram algo movendo"-se \textit{a distância}''. E com crase, se a
distância é especificada: ``O portão ficava \textit{à distância} de 4 m''. 
Sugere"-se, porém, mesmo no primeiro caso, usar da crase, quando a sua
falta comprometer de algum modo a clareza da frase: ``A
sentinela vigia à distância''. \textsc{houaiss} (s.d)} no mundo
positivo da indústria moderna.

Tão pouco medeia do Rio a S. Paulo! Por que não conseguiremos
enxergar de um a outro cabo, em linha tão curta? Tentemos. Vejamos.
Estendamos as mãos, entre os dois pontos que a limitam. Deste àquele já
se estabeleceu a corrente. Rápida como o pensamento, corre a emanação
magnética desta extremidade à oposta. As mãos já se encontraram. Já num
aperto se confundiram as mãos, que se procuravam. Já, num amplexo de
todos, nos abraçamos uns aos outros. Em S. Paulo estamos. Conversemos,
amigos, de presença a presença.

Entrelaçando a colação do vosso grau com a comemoração jubilar da
minha, e dando"-me a honra de vos ser eu paraninfo, urdis, desta
maneira, no ingresso à carreira que adotastes, um como vínculo sagrado
entre a vossa existência intelectual, que se enceta, e a do vosso
padrinho em letras, que se acerca do seu termo. Do ocaso
de uma surde\footnote{ \textit{surde} (do vebo surdir): emerge,
resulta.} o arrebol da outra.

Mercê, porém, de circunstâncias inopinadas, com o encerro do meu
meio século de trabalho na jurisprudência se ajusta o remate dos meus
cinquenta anos de serviços à nação. Já o jurista começava a olhar com
os primeiros toques de saudade para o instrumento, que, há dez lustros,
lhe vibra entre os dedos, lidando pelo direito, quando a consciência
lhe mandou que despisse as modestas armas da sua luta, provadamente
inútil, pela grandeza da pátria e suas liberdades, no parlamento.

Essa remoção da metade total de um século de vida laboriosa para o
desentulho do tempo não podia consumar sem abalo sensível numa
existência repentinamente decepada. Mas a comoção foi salutar; porque o
espírito encontrou logo seu equilíbrio na convicção de que, afinal, me
chegava eu a conhecer a mim mesmo, reconhecendo a escassez de minhas
reservas de energia, para acomodar o ambiente da época às minhas ideias
de reconciliação da política nacional com o regímen republicano.

Era presunção, era temeridade, era inconsciência insistir na
insana pretensão da minha fraqueza. Só um predestinado poderia arrostar
empresa tamanha. Desde 1892 me empenhava eu em lutar com esses mares e
ventos. Não os venci. Venceram"-me eles a mim. Era natural. Deus nos dá
sempre mais do que merecemos. Já me não era pouco a graça (pela qual
erguia as mãos ao céu) de abrir os olhos à realidade evidente da minha
impotência, e poder recolher as velas, navegante desenganado, antes que
o naufrágio me arrancasse das mãos a bandeira sagrada.

Tenho o consolo de haver dado a meu país tudo o que me estava ao
alcance: a desambição, a pureza, a sinceridade, os excessos de
atividade incansável, com que, desde os bancos acadêmicos, o servi, e o
tenho servido até hoje.

Por isso me saí da longa odisseia sem créditos de
Ulisses.\footnote{ Ulisses foi o herói do poema épico grego                           
\textit{Odisseia}, de Homero.  Sua figura transcendeu o âmbito da
mitologia grega e se converteu em símbolo da capacidade do homem para
superar as adversidades.} Mas, se o não soube imitar nas artes
medrançosas de político fértil em meios e manhas, em compensação tudo
envidei por inculcar ao povo os costumes da liberdade e à república as
leis do bom governo, que prosperam os Estados, moralizam as sociedades,
e honram as nações.

Preguei, demonstrei, honrei a verdade eleitoral, a verdade
constitucional, a verdade republicana. Pobres clientes estas, entre
nós, sem armas, nem oiro, nem consideração, mal achavam, em uma
nacionalidade esmorecida e indiferente, nos títulos rotos do seu
direito, com que habilitar o mísero advogado a sustentar"-lhes com alma,
com dignidade, com sobrançaria, as desprezadas reivindicações. As três
verdades não podiam alcançar melhor sentença no tribunal da corrupção
política do que o Deus vivo no de Pilatos.

Quem por uma causa destas combateu, abraçado com ela, em vinte e
oito anos da sua Via Dolorosa,\footnote{ Via Dolorosa é uma
rua na cidade velha de Jerusalém, começa na Porta de
Santo Estevão e percorre a parte ocidental da
Cidade Velha de Jerusalém, terminando na Igreja
do Santo Sepulcro. De acordo com a tradição
cristã foi por este caminho que
Jesus Cristo carregou a
cruz. A rua possui nove das catorze estações da
cruz. As 5 últimas estações estão no interior da
Igreja do Santo
Sepulcro.} não se pode ter habituado a maldizer, senão a
perdoar, nem a descrer, senão a esperar. Descrer da cegueira humana,
sim; mas da Providência, fatal nas suas soluções, bem que (ao parecer)
tarda nos seus passos, isso nunca.

Assim que a bênção do paraninfo não traz fel. Não lhe
encontrareis no fundo nem rancor, nem azedume, nem despeito. \textit{Os
maus} só lhe inspiram tristeza e piedade. Só \textit{o mal} é o que o
inflama em ódio. Porque o ódio ao mal é amor do bem, e a ira contra o
mal, entusiasmo divino. Vede Jesus despejando os vendilhões do tempo,
ou Jesus provando a esponja amarga no Gólgota.\footnote{ Calvário (em
aramaico Gólgota) é o nome dado à colina que na época de
Cristo ficava fora da cidade de
Jerusalém, onde Jesus
foi crucificado. \textit{Calvaria} em latim,
(\textit{Kraniou Topos}) transliterado em grego e \textit{Gûlgaltâ} 
em transliteração do aramaico. O termo significa “caveira”,
referindo"-se a uma colina ou platô que contém uma pilha de crânios ou a
um acidente geográfico que se assemelha a um crânio.} Não são o
mesmo Cristo, esse ensanguentado Jesus do Calvário e aqueloutro, o
Jesus iroso, o Jesus armado, o Jesus do látego inexorável? Não serão um
só Jesus, o que morre pelos bons, e o que açoita os maus?

O padre Manuel Bernardes\footnote{ Manuel Bernardes (1644--1710), padre e
escritor português, foi um dos mais notáveis exemplos da
chamada prosa acadêmica da literatura portuguesa. Sua obra é de caráter
moral, revelando um erudito humanista. Nela
predomina a alegoria, como, por exemplo, o diálogo entre a alma, a
inteligência e a memória. Escreveu num estilo fluente, evitando
repetições e usando períodos longos, como em latim. Seus livros
principais são: \textit{Luz e calor} (1696), um tratado de vida espiritual, \textit{Nova
floresta} (5 vols. 1706, 1708, 1711, 1726 e 1728), \textit{Tratado dos últimos
fins do homem} (1726), \textit{Meditações sobre os principais mistérios da
Virgem Maria} (1737) e \textit{Estímulos práticos para seguir o Bem e fugir o
mal} (1730).} pregava, numa das suas Silvas:\footnote{  Referência
ao livro \textit{Nova floresta ou silva de vários apotegmas} de Manuel
Bernardes.}

\begin{hedraquote}
Bem pode haver ira, sem haver pecado. \textit{Irascimini,
et nolite peccare.} E às vezes poderá haver pecado, se não houver ira;
porquanto a paciência e silêncio fomenta a negligência dos maus, e
tenta a perseverança dos bons. \textit{Qui cum causa non irascitur,
peccat (diz um padre)};\textit{ patientia enim irrationabilis vitia
seminat, negligentiam nutrit, et non solum malos, sed etiam bonos
invitat ad malum}. Nem o irar"-se nestes termos é
contra a mansidão; porque esta virtude compreende dois atos: um é
reprimir a ira, quando é desordenada; outro, excitá"-la, quando convém.
A ira se compara ao cão, que ao ladrão ladra, ao senhor festeja, ao
hóspede nem festeja, nem ladra: e sempre faz o seu ofício. E assim quem
se agasta nas ocasiões, e contra as pessoas, que convém agastar"-se, bem
pode, com tudo isso, ser verdadeiramente manso. \textit{Qui igitur
(disse o Filósofo) ad quae oportet, et quibus oportet, irascitur,
laudatur, esseque is mansuetus potest.}\footnote{
\textit{Luz e Carlor}, 1ª ed., 1696, pp. 271--272 \S \textsc{xviii}.}
\end{hedraquote}


Nem toda ira, pois, é maldade; porque a ira, se, as mais das
vezes, rebenta agressiva e daninha, muitas outras, oportuna e
necessária, constitui o específico da cura. Ora deriva da tentação
infernal, ora de inspiração religiosa. Comumente se acende em
sentimentos desumanos e paixões cruéis; mas não raro flameja do amor
santo e da verdadeira caridade. Quando um braveja contra o bem, que não
entende, ou que o contraria, é ódio iroso, ou ira odienta. Quando
verbera o escândalo, a brutalidade, ou o orgulho, não é agrestia rude,
mas exaltação virtuosa; não é soberba, que explode, mas indignação que
ilumina; não é raiva desaçaimada, mas correção fraterna. Então, não
somente não peca o que se irar, mas pecará, não se irando. Cólera será;
mas cólera da mansuetude, cólera da justiça, cólera que reflete a de
Deus, face também celeste do amor, da misericórdia e da santidade.

Dela esfuzilam centelhas, em que se abrasa, por vezes, o apóstolo,
o sacerdote, o pai, o amigo, o orador, o magistrado. Essas faúlhas da
substância divina atravessam o púlpito, a cátedra, a tribuna, o
rostro,\footnote{ Tribuna dos oradores romanos, ornada com proas de
navios conquistados aos inimigos.} a imprensa, quando se debatem, ante
o país, ou o mundo, as grandes causas humanas, as grandes causas
nacionais, as grandes causas populares, as grandes causas sociais, as
grandes causas da consciência religiosa. Então a palavra se eletriza,
brame, lampeja, atroa, fulmina. Descargas sobre descargas rasgam o ar,
incendeiam o horizonte, cruzam em raios o espaço. É a hora das
responsabilidades, a hora da conta e do castigo, a hora das apóstrofes,
imprecações e anátemas, quando a voz do homem reboa como o canhão, a
arena dos combates da eloquência estremece como campo de batalha, e as
siderações da verdade, que estala sobre as cabeças dos culpados,
revolvem o chão, coberto de vítimas e destroços incruentos, com abalos
de terremoto. Ei"-la aí a cólera santa! Eis a ira divina!

Quem, senão ela, há de expulsar do templo o renegado, o blasfemo,
o profanador, o simoníaco? quem, senão ela, exterminar da ciência o
apedeuta, o plagiário, o charlatão? quem, senão ela, banir da sociedade
o imoral, o corruptor, o libertino? quem, senão ela, varrer dos
serviços do Estado o prevaricador, o concussionário e o ladrão público?
quem, senão ela, precipitar do governo o negocismo, a prostituição
política, ou a tirania? quem, senão ela, arrancar a defesa da pátria à
cobardia, à inconfidência ou à traição? Quem, senão ela, ela a cólera
do celeste inimigo dos vendilhões e dos hipócritas? a cólera do justo,
crucifixo entre ladrões? a cólera do Verbo da verdade, negado pelo
poder da mentira? a cólera da santidade suprema, justiçada pela mais
sacrílega das opressões?

Todos os que nos dessedentamos nessa fonte, os que nos saciamos
desse pão, os que adoramos esse ideal, nela vamos buscar a chama
incorruptível. É dela que, ao espetáculo ímpio do mal tripudiante sobre
os reveses do bem, rebenta em labaredas a indignação, golfa a cólera em
borbotões das fráguas da consciência, e a palavra saí, rechinando,
esbraseando, chispando como o metal candente dos seios da fornalha.

Esse metal nobre, porém, na incandescência da sua ebulição, não
deixa escória. Pode crestar os lábios, que atravessa. Poderá inflamar
por momentos o irritado coração, de onde jorra. Mas não o degenera, não
o macula, não o resseca, não o caleja, não o endurece; e, no fundo, são
da urna onde tumultuavam essas procelas, e donde borbotam essas
erupções, não assenta um rancor, uma inimizade, uma vingança. As
reações da luta cessam, e fica, de envolta com o aborrecimento ao mal,
o relevamento dos males padecidos.

Nest'alma, tantas vezes ferida e traspassada
tantas vezes, nem de agressões, nem de infamações, nem de preterições,
nem de ingratidões, nem de perseguições, nem de traições, nem de
expatriações perdura o menor rasto, a menor ideia de revindita. Deus me
é testemunha de que tudo tenho perdoado. E, quando lhe digo, na oração
dominical: ``Perdoai"-nos, Senhor, as nossas dívidas, assim
como nós perdoamos aos nossos devedores'',\footnote{ Até
meados dos anos sessenta, assim era enunciado este trecho tradicional
do  Pai"-Nosso: ``Perdoai as nossas dívidas, assim como nós perdoamos aos
nossos devedores''. Hoje, ``Perdoai"-nos as nossas ofensas assim como nós
perdoamos a quem nos tem ofendido''. Esclarece"-se que existem dois
Pais"-Nossos diferentes na Bíblia: um no Evangelho de Mateus, e outro no
Evangelho de Lucas. E o que a Igreja tinha feito ao alterar a frase
antes citada, fora simplesmente passar da versão de Mateus para a de
Lucas. Embora em latim continuemos a dizer \textit{debita} e
\textit{debitores} (“dívidas” e “devedores”, respectivamente).}
julgo não lhe estar mentindo; e a consciência me atesta que, até onde
alcance a imperfeição humana, tenho\footnote{ Na edição \textsc{ol} (1921) está
\textit{tendo}. } conseguido, e consigo todos os dias obedecer ao
sublime mandamento. Assim me perdoem, também, os a quem tenho agravado,
os com quem houver sido injusto, violento, intolerante, maligno, ou
descaridoso.

Estou"-vos abrindo o livro da minha vida. Se me não quiserdes
aceitar como expressão fiel da realidade esta versão rigorosa de uma
das suas páginas, com que mais me consolo, recebei"-a, ao menos, como
ato de fé, ou como conselho de pai a filhos, quando não como o
testamento de uma carreira, que poderá ter discrepado, muitas vezes, do
bem, mas sempre o evangelizou com entusiasmo, o procurou com fervor, e
o adorou com sinceridade.

Desde que o tempo começou, lento lento, a me decantar o espírito
do sedimento das paixões, com que o verdor dos anos e o amargor das
lutas o enturbavam, entrando eu a considerar com filosofia nas leis da
natureza humana, fui sentindo quanto esta necessita da contradição,
como a lima dos sofrimentos a melhora, a que ponto o acerbo das
provações a expurga, a tempera, a nobilita, a regenera. Então vim a
perceber vivamente que imensa dívida cada criatura da nossa espécie
deve aos seus inimigos e desfortunas. Por mais desagrestes\footnote{
\textit{desagrestes}: muito agrestes.  O prefixo \textit{des-} aqui
significa aumento, reforço, intensidade.} que sejam os
contratempos da sorte e as maldades dos homens, raro nos causam mal
tamanho, que nos não façam ainda maior bem. Ai de nós, se esta
purificação gradual, que nos deparam as vicissitudes cruéis da
existência, não encontrasse a colaboração providencial da
fortuna adversa e
dos nossos desafetos. Ninguém mete em conta o serviço contínuo, de que
lhes está em obrigação.

Diríeis, até, que, mandando"-nos amar aos nossos inimigos, em boa
parte nos quis o divino legislador entremostrar o muito, de que eles
nos são credores. A caridade com os que nos malquerem, e os que nos
malfazem, não é, em bem larga escala, senão pago dos benefícios, que,
mal a seu grado, mas muito deveras, eles nos granjeiam.

Destarte, não equivocaremos a aparência com a realidade, se, nos
dissabores que malquerentes e malfazentes nos propinam, discernirmos a
quota de lucro, com que eles, não levando em tal o sentido, quase
sempre nos favorecem. Quanto é pela minha parte, o melhor do que sou,
bem assim o melhor do que me acontece, frequentemente acaba o tempo
convencendo"-me de que não me vem das doçuras da fortuna propícia, ou da
verdadeira amizade, senão sim que o devo, principalmente, às
maquinações dos malévolos e às contradições da sorte madrasta. Que
seria, hoje, de mim, se o veto dos meus adversários, sistemático e
pertinaz, me não houvesse poupado aos tremendos riscos dessas alturas,
``alturas de Satanás'', como as de que fala o
Apocalipse, em que tantos se têm perdido,
mas a que tantas vezes me tem tentado exalçar o voto dos meus amigos?
Amigos e inimigos estão, amiúde, em posições trocadas. Uns nos querem
mal, e fazem"-nos bem. Outros nos almejam o bem, e nos trazem o mal.

Não poucas vezes, pois, razão é lastimar o zelo dos amigos, e
agradecer a malevolência dos opositores. Estes nos salvam, quando
aqueles nos extraviam. De sorte que, no perdoar aos inimigos, muita vez
não vai semente caridade cristã, senão também justiça ordinária e
reconhecimento humano. E, ainda quando, aos olhos de mundo, como aos do
nosso juízo descaminhado, tenham logrado a nossa desgraça, bem pode ser
que, aos olhos da filosofia, aos da crença e aos da verdade suprema,
não nos hajam contribuído senão para a felicidade.

Este, senhores, será um saber vulgar, um saber rasteiro,
``um saber só de experiência feito''.\footnote{ Camões,
\textit{Os lusíadas}, \textsc{iv}, 94.}


Não é o saber da ciência, que se libra acima das nuvens, e alteia
o voo soberbo, além das regiões siderais, até aos páramos indevassáveis
do infinito. Mas, ainda assim, este saber fácil mereceu a
Camões o ter a sua legenda insculpida
em versos imortais; quanto mais a nós outros, \textit{bichos da terra
tão pequenos},\footnote{ Referência ao último  verso do canto \textsc{i} de
\textit{Os lusíadas:} \textit{contra um bicho da terra tão
pequeno.}} a ninharia de ocupar divagações, como estas, de um
dia, folhas de árvore morta, que, talvez, não vinguem ao de amanhã.

Da ciência estamos aqui numa catedral. Não cabia em um velho
catecúmeno vir ensinar a religião aos seus bispos e pontífices, nem aos
que agora nela recebem as ordens do seu sacerdócio. E hoje é féria,
ensejo para tréguas ao trabalho ordinário, quase dia santo. Labutastes
a semana toda, o vosso curso de cinco anos, com teorias, hipóteses e
sistemas, com princípios, teses e demonstrações, com leis, códigos e
jurisprudências, com expositores, intérpretes e escolas. Chegou o
momento de vos assentardes, mão por mão,\footnote{ \textit{mão por mão}: 
intimamente, a sós.} com os vossos sentimentos, de vos
pordes à fala com a vossa consciência, de praticardes familiarmente com
os vossos afetos, esperanças e propósitos.

Eis ao que vem o padrinho, o velho, o abendiçoador, carregado de
anos e tradições, versado nas longas lições do tempo, mestre de
humildade, arrependimento e desconfiança, nulo entre os grandes da
inteligência, grande entre os experimentados na fraqueza humana. Que se
feche, pois, alguns momentos, o livro da ciência; e folheemos juntos o
da experiência. Desaliviemo"-nos do saber humano, carga formidável, e
voltemo"-nos uma hora para este outro, leve, comezinho, desalinhado,
conversável, seguro, sem altitudes, nem despenhadeiros.

Ninguém, senhores meus, que empreenda uma jornada extraordinária,
primeiro que meta o pé na estrada, se esquecerá de entrar em conta com
as suas forças, por saber se a levarão ao cabo. Mas, na grande viagem,
na viagem de trânsito deste a outro mundo, não há \textit{possa, ou não
possa}, não há querer, ou não querer. A vida não tem mais que duas
portas: uma de entrar, pelo nascimento; outra de sair, pela morte.
Ninguém, cabendo"-lhe a vez, se poderá furtar à entrada. Ninguém, desde
que entrou, em lhe chegando o turno, se conseguirá evadir à saída. E,
de um ao outro extremo, vai o caminho, longo, ou breve, ninguém o sabe,
entre cujos termos fatais se debate o homem, pesaroso de que entrasse,
receoso da hora em que saia, cativo de um e outro mistério, que lhe
confinam a passagem terrestre.

Não há nada mais trágico do que a fatalidade inexorável deste
destino, cuja rapidez ainda lhe agrava a severidade.

Em tão breve trajeto cada um há de acabar a sua tarefa. Com que
elementos? Com os que herdou, e os que cria. Aqueles são a parte da
natureza. Estes, a do trabalho.

A parte da natureza varia ao infinito. Não há, no universo, duas
coisas iguais. Muitas se parecem umas às outras. Mas todas entre si
diversificam. Os ramos de uma só árvore, as folhas da mesma planta, os
traços da polpa de um dedo humano, as gotas do mesmo fluido, os
argueiros do mesmo pó, as raias do espectro de um só raio solar ou
estelar. Tudo assim, desde os astros, no céu, até os micróbios no
sangue, desde as nebulosas no espaço, até aos aljôfares do rocio na
relva dos prados.

A regra da igualdade não consiste senão em quinhoar desigualmente
aos desiguais, na medida em que se desigualam. Nesta desigualdade
social, proporcionada à desigualdade natural, é que se acha a
verdadeira lei da igualdade. O mais são desvarios da inveja, do
orgulho, ou da loucura. Tratar com desigualdade a iguais, ou a
desiguais com igualdade, seria desigualdade flagrante, e não igualdade
real. Os apetites humanos conceberam inverter a norma universal da
criação, pretendendo, não dar a cada um, na razão do que vale, mas
atribuir o mesmo a todos, como se todos se equivalessem.

Esta blasfêmia contra a razão e a fé, contra a civilização e a
humanidade, é a filosofia da miséria, proclamada em nome dos direitos
do trabalho; e, executada, não faria senão inaugurar, em vez da
supremacia do trabalho, a organização da miséria.

Mas, se a sociedade não pode igualar os que a natureza criou
desiguais, cada um, nos limites da sua energia moral, pode reagir sobre
as desigualdades nativas, pela educação, atividade e perseverança. Tal
a missão do trabalho.

Os portentos, de que esta força é capaz, ninguém os calcula. Suas
vitórias na reconstituição da criatura maldotada só se comparam às da oração.

Oração e trabalho são os recursos mais poderosos na criação moral
do homem. A oração é o íntimo sublimar"-se d'alma pelo
contato com Deus. O trabalho é o inteirar, o desenvolver, o apurar das
energias do corpo e do espírito, mediante a ação contínua de cada um
sobre si mesmo e sobre o mundo onde labutamos.

O indivíduo que trabalha, acerca"-se continuamente do autor de
todas as coisas, tomando na sua obra uma parte, de que depende também a
dele. O Criador começa, e a criatura acaba a criação de si própria.

Quem quer, pois, que trabalhe, está em oração ao Senhor. Oração
pelos atos, ela emparelha com a oração pelo culto. Nem pode ser que uma
ande verdadeiramente sem a outra. Não é trabalho digno de tal nome o do
mau; porque a malícia do trabalhador o contamina. Não é oração
aceitável a do ocioso; porque a ociosidade a dessagra. Mas, quando o
trabalho se junta à oração, e a oração com o trabalho, a segunda
criação do homem, a criação do homem pelo homem, semelha às vezes, em
maravilhas, à criação do homem pelo divino Criador.

Ninguém desanime, pois, de que o berço lhe não fosse generoso,
ninguém se creia malfadado, por lhe minguarem de nascença haveres e
qualidades. Em tudo isso não há surpresas, que se não possam esperar da
tenacidade e santidade no trabalho. Quem não conhece a história do
padre Suarez, o autor do tratado \textit{Das Leis e de Deus Legislador}
\textit{(De Legibus ac Deo Legislatore)}, monumento jurídico, a que os
trezentos anos de sua idade ainda não gastaram o conceito de honra das
letras castelhanas? De cinquenta aspirantes, que, em 1564, solicitavam,
em Salamanca, ingresso à Companhia de Jesus, esse foi o único rejeitado, por curto de
entendimento e revesso ao ensino. Admitido, todavia, a insistências
suas, com a nota de \textit{indiferente}, embora primasse entre os mais
aplicados, tudo lhe eram, no estudo, espessas trevas. Não avançava um
passo, Afinal, por consenso de todos, passava por invencível a sua
incapacidade. Confessou"-a, por fim, ele mesmo, requerendo ao reitor, o
célebre padre Martin Gutierrez, que o escusasse da vida escolar, e o
entregasse aos misteres corporais de irmão coadjutor. Gutierrez
animou"-o a orar, persistir, e esperar. De repente se lhe alagou de
claridade a inteligência. Mergulhou"-se, então, cada vez mais no estudo;
e daí, com estupenda mudança, começa a deixar ver o a que era destinada
aquela extraordinária cabeça, até esse tempo submersa em densa escuridade.

Já é mestre insigne, já encarna todo o saber da renascença
teológica, em que brilham as letras de Espanha.\footnote{ Rui, seguindo
os clássicos da língua, dispensa o artigo antes do nome de países e
regiões familiarmente ligados a Portugal, como Espanha, França,
Inglaterra, África. Lembrem"-se as suas \textit{Cartas de Inglaterra.}}
Sucessivamente ilustra as cadeiras de filosofia, teologia e
cânones nas mais famosas
universidades europeias: em Segóvia, em Valhadolid, em Roma, em
Alcalá, em Salamanca, em Ávila, em Coimbra. 
Nos seus setenta anos de vida, professa as ciências
teológicas durante quarenta e sete, escreve cerca de duzentos volumes,
e morre comparado com Santo Agostinho e S.~Tomás, abaixo de quem houve quem
o considerasse ``o maior engenho, que tem tido a
igreja'';\footnote{ P\textsuperscript{e} Francisco Suarez: \textit{Tratado
de las Leyes y de Dios Legislador}. Ed.~de Madrid, 1918. Tomo I, p.
\textsc{xxxvii}.} sendo tal a sua nomeada, ainda entre os protestantes,
que deste jesuíta, como teólogo e filósofo, chegou a dizer
Grocio\footnote{ Hugo Grócio (Delft, 1583---Rostock, 1645),
ou Huig de Groot ou ainda Grotius, foi um
jurista dos Países Baixos. Era filho de Jan de Groot, 
curador da Universidade de Leyden. Contribuiu sensivelmente 
para os fundamentos do Direito Internacional e para a teoria do
Direito Natural. Sua obra mais conhecida é \textit{De iure belli ac pacis} 
(Das leis de guerra e paz, 1625), no qual aparece o conceito de guerra 
justa e do Direito Natural. Foi também filósofo, dramaturgo e poeta.} 
que ``apenas havia quem o igualasse''.

Já vedes que ao trabalho nada é impossível. Dele não há extremos,
que não sejam de esperar. Com ele nada pode haver, de que desesperar.

Mas, do século \textsc{xvi} ao século \textsc{xx}, o que as ciências cresceram, é
incomensurável. Entre o currículo da teologia e filosofia no primeiro,
e o programa de um curso jurídico, no segundo, a distância é infinita.
Sobre os mestres, os sábios e os estudantes de agora pesam montanhas e
montanhas mais de questões, problemas e estudos que quantos, há três ou
quatro séculos, se abrangiam no saber humano.

O trabalho, pois, vos há de bater à porta dia e noite; e nunca vos
negueis às suas visitas, se quereis honrar vossa vocação, e estais
dispostos a cavar nos veios de vossa natureza, até dardes com os
tesoiros, que aí vos haja reservado, com ânimo benigno, a dadivosa
Providência. Ouvistes o aldrabar\footnote{ \textit{aldrabar}: Pôr
aldraba em; fechar com aldraba (trinco, ferrolho).} da mão oculta, que vos chama
ao estudo? Abri, abri, sem detença. Nem por vir muito cedo, lho leveis
a mal, lho tenhais à conta de importuna. Quanto mais
matutinas essas interrupções do vosso dormir, mais lhas deveis agradecer.

O amanhecer do trabalho há de antecipar"-se ao amanhecer do dia.
Não vos fieis muito de quem esperta já sol nascente, ou sol
nado.\footnote{ Particípio passado irregular de \textit{nascer}:
nascido, nato.} Curtos se fizeram os dias, para que nós os
dobrássemos, madrugando. Experimentai, e vereis quanto vai do deitar
tarde ao acordar cedo. Sobre a noite o cérebro pende ao sono.
Antemanhã, tende a despertar.

Não invertais a economia do nosso organismo: não troqueis a noite
pelo dia, dedicando este à cama, e aquela às distrações. O que se
esperdiça para o trabalho com as noitadas inúteis, não se lhe recobra
com as manhãs de extemporâneo dormir, ou as tardes de cansado labutar.
A ciência, zelosa do escasso tempo que nos deixa a vida, não dá lugar
aos tresnoites libertinos. Nem a cabeça já exausta, ou estafada nos
prazeres, tem onde caiba o inquirir, o revolver, o meditar do estudo.

Os próprios estudiosos desacertam, quando, iludidos por um hábito
de inversão, antepõem o trabalho, que entra pela noite, ao que precede
o dia. A natureza nos está mostrando com exemplos a verdade. Toda ela,
nos viventes, ao anoitecer, inclina para o sono. A esta lição geral só
abrem triste exceção os animais sinistros e os carniceiros. Mas, quando
se avizinha o volver da luz, muito antes que ela arraie a natureza, e
ainda primeiro que alvoreça no firmamento, já rompeu na terra em
cânticos a alvorada, já se orquestram de harmonias e melodias campos e
selvas, já o galo, não o galo triste do luar dos sertões do nosso
Catulo,\footnote{ Catulo da Paixão Cearense (São
Luís do Maranhão, 1863---Rio de Janeiro, 1946) foi
teatrólogo, poeta, músico, compositor e
cantor brasileiro. Mudou"-se para o Rio em 1880. Conheceu vários
chorões  da época, como Anacleto de Medeiros  e
Viriato Figueira da Silva, quando se iniciou na
música. Suas mais famosas composições são \textit{Luar
do Sertão}, de 1908, que na opinião de Pedro Lessa  
é o hino nacional do sertanejo brasileiro, e 
\textit{Flor amorosa} (s. d.). Também foi o
responsável pela reabilitação do violão nos salões da alta sociedade
carioca e pela reforma da modinha.} mas o galo festivo das
madrugadas, retine ao longe a estridência dos seus clarins, 
vibrantes de jubilosa alegria.

Ouvi, no poema de Jó, a voz do Senhor, perguntando a seu
servo, onde estava, quando o louvavam as estrelas da manhã: \textit{Ubi
eras cum me laudarent simul astra matutina?} E que têm mais as estrelas
da manhã, dizia um grande escritor nosso, ``que têm mais as estrelas da manhã
que as da tarde, ou as da noite, para fazer Deus mais caso do louvor de
umas que das outras? não é ele o Senhor do tempo, que deve ser louvado
a todo o tempo, não só da luz, mas também das trevas? Assim é; porém as
estrelas da manhã têm esta vantagem que madrugam, antecipam"-se, e
despertam aos outros, que se levantem a servir a Deus. Pois disto é que
Deus se honra, e agrada em presença de Jó''.\footnote{\ P\textsuperscript{e} M.~Bernardes: 
\textit{Sermões e práticas}, 1ª ed., de 1762. Parte \textsc{i}, p.~297.}

Tomai exemplo, estudantes e doutores, tomai exemplo das estrelas
da manhã, e gozareis das mesmas vantagens: não só a de levantardes mais
cedo a Deus a oração do trabalho, mas a de antecederdes aos demais,
logrando mais para vós mesmos, e estimulando os outros a que vos
rivalizem no ganho bendito.

Há estudar, e estudar. Há trabalhar, e trabalhar. Desde que o
mundo é mundo, se vem dizendo que o homem nasce para o trabalho:
\textit{Homo nascitur ad laborem}.\footnote{ Jó, \textsc{v}, 7.} 
Mas o trabalhar é como o semear, onde tudo vai muito das sazões,
dos dias e das horas. O cérebro, cansado e seco do laborar diurno, não
acolhe bem a semente: não a recebe fresco e de bom grado, como a terra
orvalhada. Nem a colheita acode tão suave às mãos do lavrador, quando o
torrão já lhe não está sorrindo entre o sereno da noite e os alvores do dia.

Assim, todos sabem que para trabalhar nascemos. Mas muitos somos
os que ignoramos certas condições, talvez as mais elementares, do
trabalho, ou, pelo menos, mui poucos os que as praticamos. Quantos
serão os que acreditem que os melhores trabalhadores sejam os melhores
madrugadores? que os mais estudiosos não sejam os que oferecem ao
estudo os sobejos do dia, mas os que o honram com as primícias da
manhã?

Dirão que tais trivialidades, cediças e corriqueiras, não são para
contempladas\footnote{ \textit{não são para contempladas}: 
não são para serem contempladas; para escutadas: para serem escutadas.} num discurso
acadêmico, nem para escutadas entre doutores, lentes e sábios. Cada um
se avém como entende, e faz o que pode. Mas eu, nisto aqui, faço ainda
o que devo. Porque, vindo pregar"-vos experiência, cumpria que relevasse
mais a que mais sobressai na minha estirada carreira de estudante.

Estudante sou. Nada mais. Mau sabedor, fraco jurista, mesquinho
advogado, pouco mais sei do que saber estudar, saber como se estuda, e
saber que tenho estudado. Nem isso mesmo sei se saberei bem. Mas, do
que tenho logrado saber, o melhor devo às manhãs e madrugadas. Muitas
lendas se têm inventado, por aí, sobre excessos da minha vida
laboriosa. Deram, nos meus progressos intelectuais, larga parte ao uso
em abuso do café e ao estímulo habitual dos pés mergulhados
n'água fria. Contos de imaginadores. Refratário sou ao
café. Nunca recorri a ele como a estimulante cerebral. Nem uma só vez
na minha vida busquei num pedilúvio o espantalho do sono.

Ao que devo, sim, o mais dos frutos do meu trabalho, a relativa
exabundância de sua fertilidade, a parte produtiva e durável da sua
safra, é às minhas madrugadas. Menino ainda, assim que entrei ao
colégio, alvidrei eu mesmo a conveniência desse costume, e daí avante o
observei, sem cessar, toda a vida. Eduquei nele o meu cérebro, a ponto
de espertar exatamente à hora, que comigo mesmo assentava, ao dormir.
Sucedia, muito amiúde, encetar eu a minha solitária banca de estudo a
uma ou às duas da antemanhã. Muitas vezes me mandava meu pai volver ao
leito; e eu fazia apenas que lhe obedecia, tornando, logo após, àquelas
amadas lucubrações, as de que me lembro com saudade mais deleitosa e
entranhável.

Tenho,\footnote{ A edição \textsc{ol} (1921) mantém \textit{Tendo}.} ainda
hoje, convicção de que nessa observância persistente está o segredo
feliz, não só das minhas primeiras vitórias no trabalho, mas de quantas
vantagens alcancei jamais levar aos meus concorrentes, em todo o andar
dos anos, até à velhice. Muito há que já não subtraio tanto às horas da
cama, para acrescentar às do estudo. Mas o sistema ainda perdura, bem
que largamente cerceado nas antigas imoderações. Até agora, nunca o sol
deu comigo deitado e, ainda hoje, um dos meus raros e modestos
desvanecimentos é o de ser grande madrugador, madrugador impenitente.

Mas, senhores, os que madrugam no ler, convém madrugarem também no
pensar. Vulgar é o ler, raro o refletir. O saber não está na ciência
alheia, que se absorve, mas, principalmente, nas ideias próprias, que
se geram dos conhecimentos absorvidos, mediante a transmutação, por que
passam, no espírito que os assimila. Um sabedor não é armário de
sabedoria armazenada, mas transformador reflexivo de aquisições digeridas.

Já se vê quanto vai do saber aparente ao saber real. O saber de
aparência crê e ostenta saber tudo. O saber de realidade, quanto mais
real, mais desconfia, assim do que vai apreendendo, como do que elabora.

Haveis de conhecer, como eu conheço, países, onde quanto menos
ciência se apurar, mais sábios florescem. Há, sim, dessas regiões, por
este mundo além. Um homem (nessas terras de promissão) que nunca se
mostrou lido ou sabido em coisa nenhuma, tido e havido é por corrente e
moente\footnote{ \textit{corrente e moente}: versado, perito; conforme
\textsc{pinto} (1954).} no que quer que seja; porque assim o aclamam as trombetas
da política, do elogio mútuo, ou dos corrilhos pessoais, e o povo
subscreve a néscia atoarda. Financeiro, administrador, estadista, chefe
de Estado, ou qualquer outro lugar de ingente situação e assustadoras
responsabilidades, é, a pedir de boca, o que se diz mão de pronto
desempenho, fórmula viva a quaisquer dificuldades, chave de todos os enigmas.

Tenham por averiguado que, onde quer que o colocarem, dará conta o
sujeito das mais árduas empresas e solução aos mais emaranhados
problemas. Se em nada se aparelhou, está em tudo e para tudo
aparelhado. Ninguém vos saberá informar por quê. Mas todo o mundo vo"-lo
dará por líquido e certo. Não aprendeu nada, e sabe tudo. Ler, não leu.
Escrever, não escreveu. Ruminar, não ruminou. Produzir, não produziu. E
um improviso onisciente, o fenômeno de que poetava Dante:

\begin{verse}
\textit{In picciol tempo gran dottor si feo}\footnote{ Em parvo tempo 
a grão sábio ascendeu.  
\textit{A divina comédia}, 2007, p. 88.} 
\end{verse}
 

A esses homens"-panaceias, a esses empreiteiros de todas as
empreitadas, a esses aviadores de todas as encomendas, se escancelam os
portões da fama, do poderio, da grandeza, e, não contentes de lhes
aplaudir entre os da terra a nulidade, ainda, quando Deus quer, a
mandam expor à admiração do estrangeiro.

Pelo contrário, os que se tem por notório e incontestável
excederem o nível da instrução ordinária, esses para nada servem. Por
quê? Porque ``sabem demais''. Sustenta"-se aí
que a competência reside, justamente, na incompetência. Vai"-se, até, ao
incrível de se inculcar ``o medo aos
preparados'', de havê"-los como cidadãos perigosos, e
ter"-se por dogma que um homem, cujos estudos passarem da craveira
vulgar, não poderia ocupar qualquer posto mais grado no governo, em
país de analfabetos. Se o povo é analfabeto, só ignorantes estarão em
termos de o governar. Nação de analfabetos, governo de analfabetos. E o
que eles, muita vez às escâncaras, e em letra redonda, por aí dizem.

Sócrates\footnote{ Sócrates nasceu em
Atenas, provavelmente no ano de 470 a.C, e
tornou"-se um dos principais pensadores da
Grécia Antiga. Podemos afirmar que Sócrates
fundou o que conhecemos hoje por filosofia
ocidental. Foi influenciado pelo conhecimento de um outro importante
filósofo grego: Anaxágoras. Seus primeiros estudos e pensamentos
discorrem sobre a essência da natureza  da alma humana.} certo dia,
numa das suas conversações, que O Primeiro Alcibíades\footnote{
Nome de um dos diálogos de Platão.  O \textit{Primeiro Alcibíades}
trata da doutrina socrática do autoconhecimento.} nos deixa escutar
ainda hoje, dava grande lição de modéstia ao interlocutor, dizendo"-lhe,
com a costumada lhaneza: ``A pior espécie de ignorância é
cuidar uma pessoa saber o que não sabe\ldots{} Tal, meu caro Alcibíades, o
teu caso. Entraste pela política, antes de a teres estudado. E não és
tu só o que te vejas nessa condição: é esta mesma a da mor parte dos
que se metem nos negócios da república. Apenas excetuo exíguo número, e
pode ser que, unicamente, a Péricles, teu tutor; porque tem cursado os filósofos''.

Vede agora os que intentais exercitar"-vos na ciência das leis, e
vir a ser seus intérpretes, se de tal jeito é que conceberíeis
sabê"-las, e executá"-las. Desse jeito; isto é: como as entendiam os
políticos da Grécia, pintada pelo mestre de Platão.

Uma vez, que Alcibíades discutia com Péricles, em palestra
registrada por Xenofonte, acertou\footnote{
\textit{acertou}: aconteceu.} de se debater o que seja \textit{lei}, e
quando exista, ou não exista.

--- Que vem a ser \textit{lei}? --- indaga Alcibíades.

--- A expressão da vontade do povo --- responde
Péricles.

--- Mas que é o que determina esse povo? O bem, ou o
mal? --- replica"-lhe o sobrinho.

--- Certo que o bem, mancebo.

--- Mas, sendo uma oligarquia quem mande, isto é, um
diminuto número de homens, serão, ainda assim, respeitáveis as lei?

--- Sem dúvida.

--- Mas, se a disposição vier de um tirano? Se ocorrer
violência, ou ilegalidade? Se o poderoso coagir o fraco? Cumprirá,
todavia, obedecer? 

Péricles hesita; mas acaba admitindo:

--- Creio que sim.

--- Mas então --- insiste Alcibíades --- o tirano, que constrange 
os cidadãos a lhe acatarem os caprichos, não será, esse sim, 
o inimigo das \textit{leis}?

--- Sim; vejo agora que errei em chamar \textit{leis} às
ordens de um tirano, costumado a mandar, sem persuadir.

--- Mas, quando um diminuto número de cidadãos impõe seus
arbítrios à multidão, daremos, ou não, a isso o nome de violência?

--- Parece"-me a mim --- concede Péricles, cada vez mais
vacilante ---, que, em caso tal, é de violência que se
trata, não \textit{de lei}.

Admitido isso, já Alcibíades triunfa:

--- Logo, quando a multidão, governando, obrigar os ricos,
sem consenso destes, não será, também, violência, e não
\textit{lei}?

Péricles não acha que responder; e a própria
razão não o acharia. Não é \textit{lei} a lei, senão quando assenta no
consentimento da maioria, já que, exigido o de todos,
\textit{desiderandum}\footnote{ \textit{desiderandum}: que se deve
desejar.} irrealizável, não haveria meio jamais de se chegar a uma lei.

Ora, senhores bacharelandos, pesai bem que vos ides consagrar à
\textit{lei}, num país onde a lei absolutamente não exprime o
consentimento \textit{da maioria}, onde são as minorias, as oligarquias
mais acanhadas, mais impopulares e menos respeitáveis, as que põem, e
dispõem, as que mandam, e desmandam em tudo; a saber: num país, onde,
verdadeiramente, \textit{não há lei}, não há, moral política ou
juridicamente falando.

Considerai, pois, nas dificuldades, em que se vão enleiar os que
professam a missão de sustentáculos e auxiliares \textit{da lei}, seus
mestres e executores.

É verdade que a execução corrige, ou atenua, muitas vezes, a
legislação de má nota. Mas, no Brasil, a \textit{lei} se deslegítima,
anula e torna \textit{inexistente}, não só pela bastardia da origem,
senão ainda pelos horrores da aplicação.

Ora, dizia S.~Paulo que boa é a lei, onde se executa legitimamente. \textit{Bona est lex,
si quis ea legitime utatur}\footnote{ S. Paulo: \textit{\textsc{i} Tim. }\textsc{i},
8.} Quereria dizer: Boa é a lei quando executada com retidão.
Isto é: boa será, em havendo no executor a virtude, que no legislador
não havia. Porque só a moderação, a inteireza e a equidade, no aplicar
das más leis, as poderiam, em certa medida, escoimar da impureza,
dureza e maldade, que encerrarem. Ou, mais lisa e claramente, se bem o
entendo, pretenderia significar o apóstolo das gentes que mais vale a
lei má, quando \textit{inexecutada}, ou mal \textit{executada} (para o
bem), que a boa lei, sofismada e não observada (contra ele).

Que extraordinário, que imensurável, que, por assim dizer,
estupendo e sobre"-humano, logo, não será, em tais condições, o papel da
justiça! Maior que o da própria legislação. Porque, se dignos são os
juízes, como parte suprema, que constituem, no executar das leis, --- em
sendo justas, lhes manterão eles a sua justiça, e, injustas, lhes
poderão moderar, se não, até, no seu tanto, corrigir a injustiça.

De nada aproveitam leis, bem se sabe, não existindo quem as ampare
contra os abusos; e o amparo sobre todos essencial é o de uma justiça
tão alta no seu poder, quanto na sua missão. ``Aí temos as
leis'', dizia o Florentino.\footnote{ Referência à Dante
Alighieri, poeta italiano (Florença, 1265---Ravena, 1321).} ``Mas quem lhes há de
ter mão?\footnote{ \textit{ter mão} ou ter mão de = segurar, tomar
cautela, parar, amparar: “\ldots{} tem mão neste cavalo, que quero ver se
posso com alguns rogos estorvar a morte daquele cavaleiro”,  Francisco
de Morais, \textit{Palmeirim de Inglaterra}, cap. 132; “Tem mão: Não
mates a teu filho”, António Vieira, Sermões, \textsc{xii}, 243; “Quando os meus
bens estavam a pique, vi tua mãe\ldots{} e tive mão do meu edifício em
ruínas\ldots{}”, Camilo, \textit{Três Irmãs}, \textsc{i}, cap. 3, 38; “Eu é que me
custa ter mão em mim”, Id., \textit{Brasileira de Prazins}, cap. \textsc{ii};
\textsc{morais silva} (1954).} Ninguém''.

\clearpage

\begin{verse}
\textit{Le leggi son, ma chi pon mano ad esse?}\\
\textit{Nullo}\footnote{ ``As leis aí estão, mas quem as vai reger?
Ninguém.'' \textit{A divina comédia}, 2007, p. 108.} 
\end{verse}

Entre nós não seria lícito responder assim tão em absoluto à
interrogação do poeta. Na Constituição brasileira, a mão que ele não
via na sua república e em sua época, a mão sustentadora das leis, aí a
temos, hoje, criada, e tão grande, que nada lhe iguala a majestade,
nada lhe rivaliza o poder. Entre as leis, aqui, entre as leis
ordinárias e a lei das leis, é a justiça quem decide, fulminando
aquelas, quando com esta colidirem.

Soberania tamanha só nas federações de molde norte"-americano cabe
ao poder judiciário, subordinado aos outros poderes nas demais formas
de governo, mas, nesta, superior a todos.

Dessas democracias, pois, o eixo é a justiça, eixo não abstrato,
não supositício, não meramente moral, mas de uma realidade profunda, e
tão seriamente implantado no mecanismo do regímen, tão praticamente
embebido através de todas as suas peças, que, falseando ele ao seu
mister, todo o sistema cairá em paralisia, desordem e subversão. Os
poderes constitucionais entrarão em conflitos insolúveis, as franquias
constitucionais ruirão por terra, e da organização constitucional, do
seu caráter, das suas funções, de suas garantias apenas restarão destroços.

Eis o de que nos há de preservar a justiça brasileira, se a
deixarem sobreviver, ainda que agredida, oscilante e malsegura, aos
outros elementos constitutivos da república, no meio das ruínas, em que
mal se conservam ligeiros traços da sua verdade.

Ora, senhores, esse poder eminencialmente necessário, vital e
salvador, tem os dois braços, nos quais aguenta a lei, em duas
instituições: a magistratura e a advocacia, tão velhas como a sociedade
humana, mas elevadas ao cem"-dobro, na vida constitucional do Brasil,
pela estupenda importância, que o novo regímen veio dar à justiça.

Meus amigos, é para colaborardes em dar existência a essas duas
instituições que hoje saís daqui habilitados. Magistrados ou advogados
sereis. São duas carreiras quase sagradas, inseparáveis uma da outra,
e, tanto uma como a outra, imensas nas dificuldades, responsabilidades
e utilidades.

Se cada um de vós meter bem a mão na consciência, certo que
tremerá da perspectiva. O tremer próprio é dos que se defrontam com as
grandes vocações, e são talhados para as desempenhar. O tremer, mas não
o descorçoar. O tremer, mas não o renunciar. O tremer, com o ousar. O
tremer, com o empreender. O tremer, com o confiar. Confiai, senhores.
Ousai. Reagi. E haveis de ser bem sucedidos. Deus, pátria, e trabalho.
Metei no regaço essas três fés, esses três amores, esses três signos
santos. E segui, com o coração puro. Não hajais medo a que a sorte vos
ludibrie. Mais pode que os seus azares a constância, a coragem e a virtude.

Idealismo? Não: experiência da vida. Não há forças, que mais a
senhoreiem, do que essas. Experimentai"-o, como eu o tenho
experimentado. Poderá ser que resigneis certas situações, como eu as
tenho resignado. Mas meramente para variar de posto, e, em vos sentindo
incapazes de uns, buscar outros, onde vos venha ao encontro o dever,
que a Providência vos haja reservado.

Encarai, jovens colegas meus, nessas duas estradas, que se vos
patenteiam. Tomai a que vos indicarem vossos pressentimentos, gostos e
explorações, no campo dessas nobres disciplinas, com que lida a ciência
das leis e a distribuição da justiça. Abraçai a que vos sentirdes
indicada pelo conhecimento de vós mesmos. Mas não primeiro que hajais
buscado na experiência de outrem um pouco da que vos é mister, e que
ainda não tendes, para eleger a melhor derrota,\footnote{ \textit{derrota}:
percurso, caminho, direção.} entre as duas que se oferecem à carta de
idoneidade, hoje obtida.

Pelo que me toca, escassamente avalio até onde, nisso, vos poderia
eu ser útil. Muito vi em cinquenta anos. Mas o que constitui a
experiência, consiste menos no ver, que no saber observar. Observar com
clareza, com desinteresse, com seleção. Observar, deduzindo, induzindo,
e generalizando, com pausa, com critério com desconfiança. Observar,
apurando, contrasteando, e guardando.

Que espécie de observador seja eu, não vo"-lo poderia dizer. Mas,
seguro, ou não, no averiguar e discernir, --- de uma qualidade, ao menos,
me posso abonar a mim mesmo: a de exato e consciencioso no expender e narrar.

Como me dilataria, porém, numa ou noutra coisa, quando tão
longamente, aqui, já me tenho excedido em abusar de vós e de mim mesmo?

Não recontarei, pois, senhores, a minha experiência, e muito menos
tentarei explaná"-la. Cingir"-me"-ei, estritamente, a falar"-vos como
falaria e mim próprio, se vós estivésseis em mim, sabendo o que tenho
experimentado, e eu me achasse em vós, tendo que resolver essa escolha.

Todo pai é conselheiro natural. Todos os pais aconselham, se bem
que nem todos possam jurar pelo valor dos seus conselhos. Os meus serão
os a que me julgo obrigado, na situação em que momentaneamente estou,
pelo vosso arbítrio, de pai espiritual dos meus afilhados em letras,
nesta solenidade.

E à magistratura que vos ides votar?

Elegeis, então, a mais eminente das profissões, a que um homem se
pode entregar neste mundo. Essa elevação me impressiona seriamente; de
modo que não sei se a comoção me não atalhará o juízo, ou tolherá o
discurso. Mas não se dirá que, em boa vontade, fiquei aquém dos meus deveres.

Serão, talvez, meras vulgaridades, tão singelas, quão sabidas, mas
ande o senso comum, a moral e o direito, associando"-se à experiência,
lhe\footnote{  \textit{lhe} = lhes, referindo"-se a “vulgaridades”.  Em
certo período da língua, até os seiscentistas, \textit{lhe} era uniforme
numericamente como o é genericamente, era singular e plural.}
nobilitam os ditames. Vulgaridades, que qualquer outro orador se
avantajaria em esmaltar de melhor linguagem, mas que, na ocasião, a mim
tocam, e no meu ensoado vernáculo hão de ser ditas. Baste, porém, que
se digam com isenção, com firmeza, com lealdade; e assim hão de ser
ditas, hoje, desta nobre tribuna.

Moços, se vos ides medir com o direito e o crime na cadeira de
juízes, começai, esquadrinhando as exigências aparentemente menos altas
dos vossos cargos, e proponde"-vos caprichar nelas com dobrado rigor;
porque, para sermos fiéis no muito, o devemos ser no pouco.

\textit{Qui fidelis est in minimo, et in majori fidelis est; et
qui in modico iniquus est, et in majori iniquus
est.}\footnote{ Lucas, \textsc{xvi}, 10.} 

Ponho exemplo, senhores. Nada se leva em menos conta, na
judicatura, a uma boa fé de ofício que o vezo de tardança nos despachos
e sentenças. Os códigos se cansam debalde em o punir. Mas a geral
habitualidade e a conivência geral o entretêm, inocentam e
universalizam. Destarte se incrementa e demanda ele em proporções
incalculáveis, chegando as causas a contar a idade por lustros, ou
décadas, em vez de anos.

Mas justiça atrasada não é justiça, senão injustiça qualificada e
manifesta. Porque a dilação ilegal nas mãos do julgador contraria o
direito escrito das partes, e, assim, as lesa no patrimônio, honra e
liberdade. Os juizes tardinheiros são culpados, que a lassidão comum
vai tolerando. Mas sua culpa tresdobra com a terrível agravante de que
o lesado não tem meio de reagir contra o delinquente poderoso, em cujas
mãos jaz a sorte do litígio pendente.

Não sejais, pois, desses magistrados, nas mãos de quem os autos
penam como as almas do purgatório, ou arrastam sonos esquecidos como as
preguiças do mato.

Não vos pareçais com esses outros juízes, que, com tabuleta de
escrupulosos, imaginam em risco a sua boa fama, se não evitarem o
contato dos pleiteantes, recebendo"-os com má sombra,\footnote{ \textit{com má sombra:} 
de cara feia.} em lugar de os
ouvir a todos com desprevenção, doçura e serenidade.

Não imiteis os que, em se lhes oferecendo o mais leve pretexto, a
si mesmos põem suspeições rebuscadas, para esquivar responsabilidades,
que seria do seu dever arrostar sem quebra de ânimo ou de confiança no
prestígio dos seus cargos.

Não sigais os que argumentam com o grave das acusações, para se
armarem de suspeita e execração contra os acusados; como se, pelo
contrário, quanto mais odiosa a acusação, não houvesse o juiz de se
precaver mais contra os acusadores, e menos perder de vista a presunção
de inocência, comum a todos os réus, enquanto não liquidada a prova e
reconhecido o delito.

Não acompanheis os que, no pretório, ou no júri, se convertem de
julgadores em verdugos, torturando o réu com severidades inoportunas,
descabidas, ou indecentes; como se todos os acusados não tivessem
direito à proteção dos seus juízes, e a lei processual, em todo o mundo
civilizado, não houvesse por sagrado o homem, sobre quem recai acusação
ainda inverificada.

Não estejais com os que agravam o rigor das leis, para se
acreditar com o nome de austeros e ilibados. Porque não há nada menos
nobre e aplausível que agenciar uma reputação malignamente obtida em
prejuízo da verdadeira inteligência dos textos legais.

Não julgueis por considerações de pessoas, ou pelas do valor das
quantias litigadas, negando as somas, que se pleiteiam, em razão da sua
grandeza, ou escolhendo, entre as partes na lide, segundo a situação
social delas, seu poderio, opulência e conspicuidade. Porque quanto
mais armados estão de tais armas os poderosos, mais inclinados é de
recear que sejam à extorsão contra os menos ajudados da fortuna; e, por
outro lado, quanto maiores são os valores demandados e maior, portanto,
a lesão arguida, mais grave iniquidade será negar a reparação, que se
demanda.

Não vos mistureis com os togados, que contraíram a doença de achar
sempre razão ao Estado, ao Governo, à Fazenda; por onde os condecora o
povo com o título de ``fazendeiros''. Essa
presunção de terem, de ordinário, razão contra o resto do mundo,
nenhuma lei a reconhece à Fazenda, ao Governo, ou ao Estado.

Antes, se admissível fosse aí qualquer presunção, havia de ser em
sentido contrário; pois essas entidades são as mais irresponsáveis, as
que mais abundam em meios de corromper, as que exercem as perseguições,
administrativas, políticas e policiais, as que, demitindo funcionários
indemissíveìs, rasgando contratos solenes, consumando lesões de toda a
ordem (por não serem os perpetradores de tais atentados os que os
pagam), acumulam, continuamente, sobre o tesoiro público terríveis
responsabilidades.

No Brasil, durante o Império, os liberais tinham por artigo do seu
programa cercear os privilégios, já espantosos, da Fazenda Nacional.
Pasmoso é que eles,\footnote{ \textit{eles}: referência aos
“privilégios”.} sob a República, se cem"-dobrem ainda, conculcando"-se,
até, a Constituição, em pontos de alto melindre, para assegurar ao
Fisco esta situação monstruosa, e que ainda haja quem, sobre todas
essas conquistas, lhe\footnote{ \textit{lhe}: referência ao “Fisco”, à
“Fazenda”.} queira granjear a de um lugar de predileções e vantagens na
consciência judiciária, no foro íntimo de cada magistrado.

Magistrados futuros, não vos deixeis contagiar de contágio tão
maligno. Não negueis jamais ao Erário, à Administração, à União, os
seus direitos. São tão invioláveis, como quaisquer outros. Mas o
direito dos mais miseráveis dos homens, o direito do mendigo, do
escravo, do criminoso, não é menos sagrado, perante a justiça, que o do
mais alto dos poderes. Antes, com os mais miseráveis é que a justiça
deve ser mais atenta, e redobrar de escrúpulo; porque são os mais
maldefendidos,\footnote{ Na edição \textsc{ol} (1921) está \textit{defendidos}.}
os que suscitam menos interesse, e os contra cujo direito conspiram a
inferioridade na condição com a míngua nos recursos.

Preservai, juízes de amanhã, preservai vossas almas juvenis desses
baixos e abomináveis sofismas. A ninguém importa mais do que à
magistratura fugir do medo, esquivar humilhações, e não conhecer
cobardia. Todo o bom magistrado tem muito de heroico em si mesmo, na
pureza imaculada e na plácida rigidez, que a nada se dobre, e de nada
se tema, senão da outra justiça, assente, cá embaixo, na consciência
das nações, e culminante, lá em cima, no juízo divino.

Não tergiverseis com as vossas responsabilidades, por mais
atribulações que vos imponham, e mais perigos a que vos exponham. Nem
receeis soberanias da terra: nem a do povo, nem a do poder. O povo é
uma torrente, que rara vez se não deixa conter pelas ações magnânimas.
A intrepidez do juiz, como a bravura do soldado, o arrebatam, e
fascinam. Os governos investem contra a justiça, provocam e
desrespeitam a tribunais; mas, por mais que lhes\footnote{ \textit{lhes
espumem contra as sentenças}: espumem contra as suas sentenças.}
espumem contra as sentenças, quando justas, não terão, por muito tempo,
a cabeça erguida em ameaça ou desobediência diante dos magistrados, que
os enfrentem com dignidade e firmeza.

Os presidentes de certas repúblicas são, às vezes, mais
intolerantes com os magistrados, quando lhes resistem, como devem, do
que os antigos monarcas absolutos. Mas, se os chefes das democracias de
tal jaez se esquecem do seu lugar, até o extremo de se haverem, quando
lhes pica o orgulho, com os juízes vitalícios e inamovíveis de hoje,
coma se haveriam com os ouvidores e desembargadores
d'El"-Rei Nosso Senhor, frágeis instrumentos nas mãos
de déspotas coroados, --- cumpre aos amesquinhados pela jactância dessas
rebeldias ter em mente que, instituindo"-os em guardas da Constituição
contra os legisladores e da lei contra os governos, esses pactos de
liberdade não os revestiram de prerrogativas ultramajestáticas, senão
para que a sua autoridade não torça às exigências de nenhuma potestade
humana.

Os tiranos e bárbaros antigos tinham, por vezes, mais compreensão
real da justiça que os civilizados e democratas de hoje. Haja vista a
história, que nos conta um pregador do século \textsc{xvii}.

``A todo o que faz pessoa de \textit{juiz}, ou
ministro'', dizia o orador sacro, ``manda
Deus que não considere na parte a razão de príncipe poderoso, ou de
pobre desvalido, senão só a razão do seu próximo\ldots{}\footnote{ Levítico,
\textsc{xix}, 15.} Bem praticou esta virtude Canuto,\footnote{ \textsc{canuto} ou
\textsc{knut}, rei da Inglaterra de 1016 a 1035 (?, \textit{c.} 994---Shaftesbury, 1035). 
No mesmo ano completou a conquista do reino inglês pelos dinamarqueses.
Canuto dividiu a Inglaterra em distritos militares governados por
condes. Em 1019, sucedeu a seu irmão no trono da Dinamarca. Conquistou
a Noruega em 1028, formando um grande império escandinavo. Na
Inglaterra, Canuto governou sabiamente e recebeu todo o apoio da
Igreja. Seu código de leis restaurou e consolidou os costumes
anglo"-saxões. Foi o primeiro governante nórdico aceito como um rei
cristão e civilizado.} rei dos Vândalos, que, mandando
justiçar uma quadrilha de salteadores, e pondo um deles embargos de que
era parente del"-Rei, respondeu: \textit{Se provar ser nosso parente,
razão é que lhe façam a forca mais alta}.\footnote{\ P.\textsuperscript{e} 
M.~Bernardes: \textit{Sermões}, Parte \textsc{i}, pp. 263--264.}

Bom é que os bárbaros tivessem deixado lições tão inesperadas às
nossas democracias. Bem poderia ser que, barbarizando"-se com esses
modelos, antepusessem elas, enfim, a justiça ao parentesco, e nos
livrassem da peste das parentelas, em matérias de governo.

Como vedes, senhores, para me não chamarem a mim revolucionário,
ando a catar minha literatura de hoje nos livros religiosos.

Outro ponto dos maiores na educação do magistrado: corar menos de
ter errado que de se não emendar. Melhor será que a sentença não erre.
Mas, se cair em erro, o pior é que se não corrija. E, se o próprio
autor do erro o remediar, tanto melhor; porque tanto mais cresce, com a
confissão, em crédito de justo, o magistrado, e tanto mais se soleniza
a reparação dada ao ofendido.

Muitas vezes, ainda, teria eu de vos dizer: Não façais, não
façais. Mas já é tempo de caçar as velas\footnote{ \textit{caçar as
velas} = recolher, colher as velas.} ao discurso. Pouco agora vos direi.

Não anteponhais o draconianismo à equidade. Dados a tão cruel
mania, ganharíeis, com razão, conceito de maus, e não de retos.

Não cultiveis sistemas, extravagâncias e singularidades. Por esse
meio lucraríeis a néscia reputação de originais; mas nunca a de sábios,
doutos, ou conscienciosos.

Não militeis em partidos, dando à política o que deveis à
imparcialidade. Dessa maneira venderíeis as almas e famas ao demônio da
ambição, da intriga e da servidão às paixões mais detestáveis.

Não cortejeis a popularidade. Não transijais com as conveniências.
Não tenhais negócios em secretarias. Não delibereis por conselheiros,
ou assessores. Não deis votos de solidariedade com outros, quem quer
que sejam. Fazendo aos colegas toda a honra, que lhes deverdes,
prestai"-lhes o crédito, a que sua dignidade houver direito; mas não
tanto que delibereis só de os ouvir, em matéria onde a confiança não
substitua a inspeção direta. Não prescindais, em suma, do conhecimento
próprio, sempre que a prova terminante vos esteja ao alcance da vista,
e se ofereça à verificação imediata do tribunal.

Por derradeiro, amigos de minha alma, por derradeiro, a última, a
melhor lição da minha experiência. De quanto no mundo tenho visto, o
resumo se abrange nestas cinco\footnote{ Cinco palavras: \textit{Não há
justiça, onde não haja Deus. } Sobre este engano, leia"-se a  página 96
do manuscrito, reproduzido na edição de \textsc{om} (1949b).  Rui havia
escrito, realmente, cinco palavras: \textit{ Não há justiça sem Deus.}
Ao substituir a frase, esqueceu"-se de recontar as palavras, agora em
número de sete.} palavras:

Não há justiça, onde não haja Deus.

Quereríeis que vo"-lo demonstrasse? Mas seria perder tempo, se já
não encontrastes a demonstração no espetáculo atual da terra, na
catástrofe da humanidade. O gênero humano afundiu"-se na matéria, e no
oceano violento da matéria flutuam, hoje, os destroços da civilização
meio destruída. Esse fatal excídio está clamando por Deus. Quando ele
tornar a nós, as nações abandonarão a guerra, e a paz, então, assomará
entre elas, a paz das leis e da justiça, que o mundo ainda não tem,
porque ainda não crê.

À justiça humana cabe, nessa regeneração, papel essencial. Assim o
saiba ela honrar. Trabalhai por isso os que abraçardes essa carreira,
com a influência da altíssima dignidade, que do seu exercício
recebereis.

Dela vos falei, da sua grandeza e dos seus deveres, com a
incompetência de quem não a tem exercido. Não tive a honra de ser
magistrado. Advogado sou, há cinquenta anos, e, já agora, morrerei
advogado. E entretanto, da advocacia no Brasil, da minha
profissão, do que nela, em experiência, acumulei, praticando"-a, que me
não será dado agora tratar. A extensão já demasiadíssima deste colóquio
em desalinho não me consentiria acréscimo tamanho. Mas que
perdereis, com tal omissão? Nada.

Na missão do advogado também se desenvolve uma espécie de
magistratura. As duas se entrelaçam, diversas nas funções, mas
idênticas no objeto e na resultante: a justiça. Com o advogado, justiça
militante. Justiça imperante, no magistrado.

Legalidade e liberdade são as tábuas da vocação do advogado. Nelas
se encerra, para ele, a síntese de todos os mandamentos. Não desertar a
justiça, nem cortejá"-la. Não lhe faltar com a fidelidade, nem lhe
recusar o conselho. Não transfugir da legalidade para a violência, nem
trocar a ordem pela anarquia. Não antepor os poderosos aos desvalidos,
nem recusar patrocínio a estes contra aqueles. Não servir sem
independência à justiça, nem quebrar da verdade ante o poder. Não
colaborar em perseguições ou atentados, nem pleitear pela iniquidade ou
imoralidade. Não se subtrair à defesa das causas impopulares, nem à das
perigosas, quando justas. Onde for apurável um grão, que seja, de
verdadeiro direito, não regatear ao atribulado o consolo do amparo
judicial. Não proceder, nas consultas, senão com a imparcialidade real
do juiz nas sentenças. Não fazer da banca balcão, ou da ciência
mercatura. Não ser baixo com os grandes, nem arrogante com os
miseráveis. Servir aos opulentos com altivez e aos indigentes com
caridade. Amar a pátria, estremecer o próximo, guardar fé em Deus, na
verdade e no bem.

Senhores, devo acabar. Quando, há cinquenta anos, saía eu daqui,
na velha Pauliceia, solitária e brumosa, como hoje saís da
transfigurada metrópole do máximo Estado brasileiro, bem outros eram
este país e todo o mundo ocidental.

O Brasil acabava de varrer do seu território a invasão
paraguaia,\footnote{ A Guerra do Paraguai foi o maior e mais sangrento
conflito armado internacional ocorrido no
continente americano. Estendeu"-se de dezembro
de 1864 a março de
1870. É também chamada Guerra da
Tríplice Aliança (\textit{Guerra de la Triple
Alianza}) na Argentina e Uruguai e de Grande Guerra, no Paraguai. O
conflito iniciou"-se quando, temeroso de que a instabilidade política no
Uruguai fosse prejudicar a estabilidade no recém"-pacificado
Rio Grande do Sul, o governo de
Dom Pedro \textsc{ii}, após um ultimato, resolveu
interferir na política interna uruguaia. A reação militar paraguaia que
se seguiu gerou então o desencadeamento da guerra. O Paraguai, que
antes da guerra atravessava uma fase marcada por grandes investimentos
econômicos em áreas específicas, encontrava"-se, então, sob o comando de
um líder considerado pouco cauteloso e inexperiente,
Francisco Solano López, que reagiu à
interferência brasileira no Uruguai declarando guerra ao Brasil.
Brasil, Argentina e Uruguai, aliados,
derrotaram o Paraguai após cinco anos de lutas
durante os quais o Brasil enviou mais de 160 mil homens à guerra. Algo
em torno 50 mil não voltaram --- alguns autores asseveram que as mortes
no caso do Brasil podem ter alcançado 60 mil se forem incluídos civis,
principalmente nas então províncias do Rio Grande do Sul e de
Mato Grosso.} e, na América do Norte,
poucos anos antes, a guerra civil\footnote{ A Guerra Civil Americana
(também conhecida em português como Guerra de
Secessão) ocorreu nos Estados Unidos da América
entre 1861 e 1865.
Consistiu na luta entre 11 estados do
Sul, com seus latifundiários e
aristocratas contra os estados do
norte, industrializado. Esta divisão é considerada
uma das causas primárias do conflito. Enquanto o norte passava por um
período de expansão econômica graças à industrialização, à proteção ao
mercado interno e à mão de obra livre e
assalariada, a economia do sul dependia da
exportação de produtos agropecuários ---
especialmente do algodão, cujas exportações
eram a principal fonte de renda destes estados. A agropecuária do sul,
por sua vez, dependia muito do uso do trabalho escravo.} limpara
da grande república o cativeiro negro, cuja agonia esteve a pique de a
soçobrar despedaçada. Eram dois prenúncios de uma alvorada, que doirava
os cimos do mundo cristão, anunciando futuras vitórias da liberdade.

Mas, ao mesmo tempo, a invasão germânica alagava terras de França,
deixando"-a violada, transpassada no coração e cruelmente mutilada, aos
olhos secos e indiferentes das outras potências e mais nações
europeias, grandes ou pequenas.

Ninguém percebeu que se estavam semeando o cativeiro e a subversão
do mundo. Daí a menos de cinquenta anos, aquela atroz exacerbação do
egoísmo político envolvia culpados e inocentes numa série de
convulsões, tal, que acreditaríeis haver"-se despejado o inferno entre
as nações da terra, dando ao inaudito fenômeno humano proporções quase
capazes de representar, na sua espantosa imensidade, um cataclismo
cósmico. Parecia estar"-se desmanchando e aniquilando o mundo. Mas era a
eterna justiça que se mostrava. Era o velho continente que principiava
e expiar a velha política, desalmada, mercantil e cínica, dos
Napoleões, Metternichs\footnote{ \textsc{metternich"-winneburg} (Klemens Wenzel Nepomuk
Lothar von Metternich, príncipe de), estadista e diplomata
austríaco (Coblença, 1773---Viena, 1859). Dominou a Europa de 1814 a
1848. Esse período é muitas vezes chamado de época de
Metternich. Metternich acreditava que a democracia e o nacionalismo
conduziriam à ruína e dirigiu os esforços da Áustria, Prússia e Rússia
no sentido de esmagar as revoltas nacionalistas em toda a Europa.} 
e Bismarcks, num ciclone de abominações 
inenarráveis, que bem depressa abrangeria, como
abrangeu, na zona das suas tremendas comoções, os outros continentes, e
deixaria revolvido o orbe inteiro em tormentas catastróficas, só Deus
sabe por quantas gerações além dos nossos dias.

O Briareu\footnote{ Na mitologia grega, Briareu (o vigoroso) era
um dos três hecatônquiros, gigantes com cem
braços e cinquenta cabeças, filhos de Gaia e
Urano. Assim como seus irmãos
ciclopes, os hecatônquiros foram aprisionados
no Tártaro por Urano, que os hostilizava desde
o nascimento. Poderosos, ajudaram Zeus a
derrotar os Titãs no episódio que ficou
conhecido como Titanomaquia.} do
inexorável mercantilismo que explorava a humanidade, o colosso do
egoísmo universal, que, durante um século, assistira impassível à
entronização dos cálculos dos governos sobre os direitos dos povos, o
reinado ímpio da ambição e da força rolava, e se desfazia, num
desmoronamento pavoroso, levando por aí a rojo impérios e dinastias,
reis, domínios, constituições e tratados. Mas a medonha intervenção dos
poderes tenebrosos do nosso destino mal estava começada. Ninguém
poderia conjeturar ainda como e quando acabará.

Neste canto da terra, o Brasil ``da hegemonia
sul"-americana'', entreluzida com a guerra do Paraguai, não
cultivava tais veleidades, ainda bem que, hoje, de todo em todo
extintas. Mas encetara uma era de aspirações jurídicas e revoluções
incruentas. Em 1888 aboliu a propriedade servil.\footnote{ A Lei Áurea
foi assinada em 13 de maio de
1888 pela Princesa
Isabel, extinguindo a escravidão no
Brasil.} Em 1889 baniu a coroa, e organizou a
república.\footnote{ A Proclamação da República Brasileira ocorreu dia
15 de novembro de 1889
no Rio de Janeiro, então capital do
Império do Brasil, na praça da Aclamação (hoje
Praça da República); quando um grupo de
militares do exército brasileiro, liderados
pelo comandante marechal Deodoro da Fonseca,
deu um golpe de Estado e depôs o imperador
D. Pedro \textsc{ii}. Institui"-se então a
República, sendo nessa data que o jurista
Rui Barbosa assinou o primeiro decreto do novo
regime, instituindo um governo provisório.} Em 1907 entrou, pela
porta de Haia, ao concerto das nações. Em 1917 alistou"-se na aliança da
civilização, para empenhar a sua responsabilidade e as suas forças
navais na guerra das guerras, em socorro do direito das gentes, cujo
código ajudara a organizar na Segunda Conferência da Paz.\footnote{ A
condição de nação neutral de que gozam os Países
Baixos fez da cidade de Haia um importante
centro para conferências e encontros internacionais. Com o tempo, a
Segunda Conferência  da Paz (1907) ficou conhecida como
Convenção de Haia ou Conferência de Haia, na qual
ocorreram uma série de acordos multilaterais entres diversas nações do
mundo, por convocação da rainha da Holanda e do czar da Rússia.
Rui Barbosa foi o representante
brasileiro nessa Convenção, sendo sua
participação louvada até hoje como uma das mais firmes, convincentes e
primorosas da diplomacia brasileira.}

Mas, de súbito, agora, um movimento desvairado parece estar"-nos
levando, empuxados de uma corrente submarina, a um recuo inexplicável.
Diríeis que o Brasil de 1921 tendesse, hoje, a repudiar o Brasil de
1917. Por quê? Porque a nossa política nos descurou dos interesses, e,
ante isso, delirando em acesso de frívolo despeito, iríamos desmentir a
excelsa tradição, tão gloriosa, quão inteligente e fecunda?

Não: senhores, não seria possível. Na resolução de 1917 o Brasil
ascendeu à elevação mais alta de toda a nossa história. Não descerá.

Amigos meus, não. Compromissos daquela natureza, daquele alcance,
daquela dignidade não se revogam. Não convertamos uma questão de futuro
em questão de relance. Não transformemos uma questão de previdência em
questão de cobiça. Não reduzamos uma imensa questão de princípios a vil
questão de interesses. Não demos de barato\footnote{ \textit{demos de
barato} = demos sem relutância ou questionamento.} a essência
eterna da justiça por uma rasteira desavença de mercadores. Não
barganhemos o nosso porvir a troco de um mesquinho prato de lentilhas.
Não arrastemos o Brasil ao escândalo de se dar em espetáculo à terra
toda como a mais fútil das nações, nação que, à distância de quatro
anos, se desdissesse de um dos mais memoráveis atos de sua vida,
trocasse de ideias, variasse de afeições, mudasse de caráter, e se
renegasse a si mesma.

Ó, senhores, não, não e não! Paladinos, ainda ontem, do direito e
da liberdade, não vamos agora mostrar os punhos contraídos aos irmãos,
com que comungávamos, há pouco, nessa verdadeira cruzada. Não percamos,
assim, o equilíbrio da dignidade, por amor de uma pendência de estreito
caráter comercial, ainda mal liquidada, sobre a qual as explicações
dadas à nação pelos seus agentes, até esta data, são inconsistentes e
furta"-cores. Não culpemos o estrangeiro das nossas decepções políticas
no exterior, antes de averiguarmos se os culpados não se achariam aqui
mesmo, entre os a quem se depara, nestas cegas agitações de ódio a
outros povos, a diversão\footnote{ A palavra ``diversão'' aqui é
empregada no sentido etimológico, latim tardio \textit{diversio, onis}
``digressão, diversão'', de \textit{divertere} ``afastar"-se, apartar"-se, ser
diferente, divergir''.} mais oportuna dos nossos erros e misérias intestinas.

O Brasil, em 1917, plantou a sua bandeira entre as da civilização
nos mares da Europa. Daí não se retrocede facilmente, sem quebra da
seriedade e do decoro, se não dos próprios interesses. Mais cuidado
tivéssemos, em tempo, com os nossos, nos conselhos da paz, se neles
quiséssemos brilhar melhor do que brilhamos nos atos da guerra, e
acabar sem contratempos ou dissabores.

Agora, o que a política e a honra nos indicam, é outra coisa. Não
busquemos o caminho de volta à situação colonial. Guardemo"-nos das
proteções internacionais. Acautelemo"-nos das invasões econômicos.
Vigiemo"-nos das potências absorventes e das raças expansionistas. Não
nos temamos tanto dos grandes impérios já saciados, quanto dos ansiosos
por se fazerem tais à custa dos povos indefesos e mal governados.
Tenhamos sentido nos ventos, que sopram de certos quadrantes do céu. O
Brasil é a mais cobiçável das presas; e, oferecida, como está, incauta,
ingênua, inerme, a todas as ambições, tem, de sobejo, com que fartar
duas ou três das mais formidáveis.

Mas o que lhe importa, é que dê começo a governar"-se a si mesmo;
porquanto nenhum dos árbitros da paz e da guerra leva em conta uma
nacionalidade adormecida e anemizada na tutela perpétua de governos,
que não escolhe. Um povo dependente no seu próprio território e nele
mesmo sujeito ao domínio de senhores não pode almejar seriamente, nem
seriamente manter a sua independência para com o estrangeiro.

Eia, senhores! Mocidade viril! Inteligência brasileira! Nobre
nação explorada! Brasil de ontem e amanhã! Dai"-nos o de hoje, que nos falta.

Mãos à obra da reivindicação de nossa perdida autonomia; mãos à
obra da nossa reconstituição interior; mãos à obra de reconciliarmos a
vida nacional com as instituições nacionais; mãos à obra de substituir
pela verdade o simulacro político da nossa existência entre as nações.
Trabalhai por essa que há de ser a salvação nossa. Mas não buscando
salvadores. Ainda vos podereis salvar a vós mesmos. Não é sonho, meus
amigos: bem sinto eu, nas pulsações do sangue, essa ressurreição
ansiada. Oxalá não se me fechem os olhos, antes de lhe ver os primeiros
indícios no horizonte. Assim o queira Deus.

