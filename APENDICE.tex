\part{Apêndice}

\chapter{O dever do advogado}


\section{Carta de Evaristo de Morais}
\bigskip

\noindent Venerando mestre e preclaro chefe
\bigskip

Para solução dum verdadeiro caso de
consciência solicito vossa \textit{palavra de ordem}, que à risca
cumprirei. Deveis ter, como toda a gente, notícia, mais ou menos
completa, do lamentável crime de que é acusado o Dr. Mendes Tavares.
Sabeis que esse moço é filiado a um agrupamento partidário que apoiou a
desastrada candidatura do Marechal Hermes. Sabeis outrossim que,
ardente admirador da vossa extraordinária mentalidade e entusiasmado
pela lição de civismo que destes em face da imposição militarista,
pus"-me decididamente ao serviço da vossa candidatura. 

Dada a suposta eleição do vosso antagonista, tenho até hoje mantido e pretendo manter
seguramente as mesmas ideias. Ocorreu todavia o triste caso a que aludi. 

O acusado Dr.~José Mendes Tavares foi meu companheiro durante
quatro anos, nos bancos escolares. Não obstante o afastamento
político, sempre tivemos relação de amistosa camaradagem. Preso,
angustiado, sem socorro imediato de amigos do seu grupo, apelou para
mim, solicitando meus serviços profissionais. 

Relutei, no princípio;
aconselhei desde logo, fosse chamado outro patrono, parecendo"-me
estar naturalmente indicado um profissional bem conhecido, hoje
deputado federal, que supus muito amigo do preso. Essa pessoa por mim
apontada escusou"-se à causa. 

A opinião pública, diante de certas
circunstâncias do fato, alarmou"-se estranhamente, chegando"-se a
considerar o acusado \textit{indigno de defesa!} Não me
parece se deva dar foros de justiça a essa ferocíssima manifestação dos
sentimentos excitados da ocasião. O acusado insiste pela prestação dos
meus humildes serviços. Eu estou de posse de elementos que em muito
diminuem, senão excluem, sua responsabilidade no caso. Recorro
respeitosamente à vossa alta autoridade e vos instituo, com
grandíssima e justificada confiança, juiz do meu proceder: devo, por
ser o acusado nosso adversário, desistir da defesa iniciada?

Prosseguindo nela, sem a menor quebra dos laços que me prendem à
bandeira do civilismo, cometo uma incorreção partidária? 

Espero de vossa generosidade resposta pronta e que sirva como sentença
inapelável, para acalmação de minha consciência. 
\bigskip

\hfill{Venerador e respeitador} 
\bigskip

\hfill{\textit{Evaristo de Morais}}

\pagebreak

\section{Carta de Rui Barbosa}
\bigskip

\noindent Rio, Vila Maria Augusta, 26 de outubro de 1911 
\bigskip

\noindent Dr. Evaristo de Morais:
\bigskip

Só agora posso acudir à sua carta de 18 do corrente, que me chegou às
mãos dois dias depois. 

Recusando"-me ao apelo, que a sua consciência
dirige à minha, cometeria eu um ato de fraqueza, que não se concilia
com a minha maneira de sentir. Quando se me impõe a solução de um caso
jurídico ou moral, não me detenho em sondar a direção das correntes
que me cercam: volto"-me para dentro de mim mesmo, e dou livremente a
minha opinião, agrade ou desagrade a minorias, ou maiorias. 

Na hipótese, tanto mais sem liberdade me acharia, para me furtar à
consulta, que me endereça, quanto ela está resolvida por 
antecedências de grande notoriedade na minha vida. Tendo assumido o
patrocínio da causa do principal acusado do crime da Avenida, cujo
protagonista militou com honras e galões na campanha do hermismo contra
a ordem civil, vê"-se o meu prezado colega, a quem tão bons serviços
deve o civilismo, diante das censuras que por isso lhe irrogam, em
presença destas questões que formula e me dirige: 

\begin{hedraquote}
Devo, \textit{por ser
o acusado nosso adversário}, desistir da defesa iniciada? 
Prosseguindo nela, sem a menor quebra dos laços que me prendem à bandeira do
civilismo, cometo uma incorreção partidária? 
\end{hedraquote}

O meu senso íntimo não hesita na resposta. 

Os partidos transpõem a órbita da sua legítima
ação, toda a vez que invadam a esfera da consciência profissional, e
pretendam contrariar a expressão do Direito. Ante essa tragédia, por
tantos lados abominável, de que foi vítima o Comandante Lopes da Cruz,
o único interesse do civilismo, a única exigência do seu programa, é
que se observem rigorosamente as condições da justiça. Civilismo quer
dizer ordem civil, ordem jurídica, a saber: governo da lei, 
contraposto ao governo do arbítrio, ao governo da força, ao governo da
espada. A espada enche hoje a política do Brasil. De instrumento de
obediência e ordem, que as nossas instituições constitucionais a
fizeram, coroou"-se em rainha e soberana. Soberana das leis. Rainha da
anarquia. Pugnando, pois, contra ela, o civilismo pugna pelo
restabelecimento da nossa Constituição, pela restauração da nossa legalidade. 

Ora, quando quer e como quer que se cometa um atentado, a
ordem legal se manifesta necessariamente por duas exigências, a
acusação e a defesa, das quais a segunda, por mais execrando que seja o
delito, não é menos especial à satisfação da moralidade pública do
que a primeira. A defesa não quer o panegírico da culpa, ou do culpado.
Sua função consiste em ser, ao lado do acusado, inocente, ou criminoso,
a voz dos seus direitos legais.

 Se a enormidade da infração reveste caracteres tais, que o sentimento
geral recue horrorizado, ou se levante contra ela em violenta
revolta, nem por isto essa voz deve emudecer. Voz do Direito no meio da
paixão pública, tão susceptível de se demasiar, às vezes pela própria
exaltação da sua nobreza, tem a missão sagrada, nesses casos, de não
consentir que a indignação degenere em ferocidade e a expiação
jurídica em extermínio cruel.

 O furor dos partidos tem posto muitas vezes os seus adversários
\textit{for a da lei}. Mas, perante a humanidade, perante o
cristianismo, perante os direitos dos povos civilizados, perantes as
normas fundamentais do nosso regímen, ninguém, por mais bárbaros que
sejam os seus atos, decai do abrigo da legalidade. Todos se acham sob
a proteção das leis, que, para os acusados, assenta na faculdade
absoluta de combaterem a acusação, articularem a defesa, e exigirem a
fidelidade à ordem processual. Esta incumbência, a tradição jurídica
das mais antigas civilizações a reservou sempre ao ministério do
advogado. A este, pois, releva honrá"-lo, não só arrebatando à
perseguição os inocentes, mas reivindicando, no julgamento dos
criminosos, a lealdade às garantias legais, a equidade, a 
imparcialidade, a humanidade.

 Esta segunda exigência da nossa vocação é a mais ingrata. Nem todos
para ela têm a precisa coragem. Nem todos se acham habilitados, para
ela, com essa intuição superior da caridade, que humaniza a repressão,
sem a desarmar. Mas os que se sentem com a força de proceder com esse
desassombro de ânimo, não podem inspirar senão simpatia às almas
bem"-formadas. 

Voltaire chamou um dia, brutalmente, à paixão pública “a
demência da canalha”. Não faltam, na história dos instintos 
malignos da multidão, no estudo instrutivo da contribuição deles para os
erros judiciários, casos de lamentável memória, que expliquem a
severidade dessa aspereza numa pena irritada contra as iniquidades da
justiça no seu tempo. No de hoje, com a opinião educada e depurada que
reina sobre os países livres, essas impressões populares têm, por via
de regra, a orientação dos grandes sentimentos. Para elas se recorre,
muitas vezes com vantagens, das sentenças dos maiores tribunais.

 Circunstâncias há, porém, ainda entre as nações mais adiantadas e
cultas, em que esses movimentos obedecem a verdadeiras alucinações
coletivas. Outras vezes a sua inspiração é justa, a sua origem
magnânima. Trata"-se de um crime detestável que acordou a cólera
popular. Mas, abrasada assim, a irritação pública entra em risco de se
descomedir. Já não enxerga a verdade com a mesma lucidez. O acusado
reveste aos seus olhos a condição de monstro sem traço de procedência
humana. A seu favor não se admite uma palavra. Contra ele tudo o que se
alega, ecoará em aplausos. 

Desde então começa a justiça a correr
perigo, e com ele surge para o sacerdócio do advogado a fase
melindrosa, cujas dificuldades poucos ousam arrostar. Faz"-se mister
resistir à impaciência dos ânimos exacerbados, que não tolera a
serenidade das formas judiciais. Em cada uma delas a sofreguidão
pública descobre um fato à impunidade. Mas é, ao contrário, o interesse
da verdade o que exige que elas se esgotem; e o advogado é o ministro
desse interesse. Trabalhando por que não faleça ao seu constituinte uma
só dessas garantias da legalidade, trabalha ele, para que não falte à
justiça nenhuma de suas garantias.

 Eis por que, seja quem for o acusado, e por mais horrenda que seja a
acusação, o patrocínio do advogado, assim entendido e exercido assim,
terá foros de meritório, e se recomendará como útil à sociedade.

 Na mais justa aversão dela incorreu a causa do infeliz, cuja defesa
aceitou o meu ilustrado colega. Aceitando"-a, pois, o eloquente advogado
corre ao encontro da impopularidade. É um rasgo de sacrifício, a que um
homem inteligente como ele se não abalançaria, sem lhe medir o
alcance, e lhe sentir o amargor. As considerações, expendidas na sua
carta, que levaram a fazê"-lo, são das mais respeitáveis. Nenhum coração
de boa têmpera lhas rejeitará. 

A cabeça esmagada pela tremenda acusação
estava indefesa. O horror da sua miséria moral lhe fechara todas as
portas. Todos os seus amigos, os seus coassociados em interesses
políticos, os companheiros de sua fortuna até o momento do crime, não
tiveram a coragem de lhe ser fiéis na desgraça. Foi então que o
abandonado se voltou para o seu adversário militante, e lhe exorou o
socorro que Deus com a sua inesgotável misericórdia nos ensina a não
negar aos maiores culpados.

 O meu prezado colega não soube repelir as mãos, que se lhe estendiam
implorativamente. A sua submissão a esse sacrifício honra aos seus
sentimentos e a nossa classe, cujos mais eminentes vultos nunca
recusaram o amparo da lei a quem quer que lho exorasse. Lachaud não
indeferiu a súplica de Troppmann, o infame e crudelíssimo autor de uma
hecatombe de oito vítimas humanas, traiçoeiramente assassinadas sob a
inspiração do roubo. 

A circunstância, cuja alegação se sublinha na sua
carta, de “ser o acusado nosso adversário”, não entra em linha de
conta, senão para lhe realçar o merecimento a esse ato de abnegação. 
Em mais de uma ocasião, na minha vida pública, não hesitei em correr ao
encontro dos meus inimigos, acusados e perseguidos, sem nem sequer
aguardar que eles mo solicitassem, provocando contra mim desabridos
rancores políticos e implacáveis campanhas de malsinação, unicamente
por se me afigurar necessário mostrar aos meus conterrâneos, com
exemplos de sensação, que acima de tudo está o serviço da justiça.
Diante dela não pode haver diferença entre amigos e adversários,
senão para lhe valermos ainda com mais presteza, quando ofendida nos
adversários do que nos amigos. 

Recuar ante a objeção de que o acusado é
“indigno de defesa”, era o que não poderia fazer o meu douto colega,
sem ignorar as leis do seu ofício, ou traí"-las. Tratando"-se de um
acusado em matéria criminal, não há causa em absoluto \textit{indigna
de defesa}. Ainda quando o crime seja de todos o mais nefando, resta
verificar a prova: e ainda quando a prova inicial seja decisiva, falta,
não só apurá"-la no cadinho dos debates judiciais, senão também vigiar
pela regularidade estrita do processo nas suas mínimas formas. Cada
uma delas constitui uma garantia, maior ou menor, da liquidação da
verdade, cujo interesse em todas se deve acatar rigorosamente. 

A este respeito não sei que haja divergências, dignas de tal nome, na ética da
nossa profissão. Zanardelli, nos seus célebres discursos aos advogados
de Brescia, acerca da advocacia, depois de estabelecer como, em
matéria civil, se faz cúmplice da iniquidade o patrono ciente e
consciente de uma causa injusta, para logo ali se dá pressa em advertir: 

\begin{hedraquote}
Em princípio, todavia, não pode ter lugar nas causas penais,
onde ainda aqueles que o advogado saiba serem culpados, \textit{não}
\textit{só podem} mas \textit{devem} ser por ele defendidos.
Mittermaier observa que os devemos defender, até no caso que deles
tenhamos, diretamente, recebido a confissão de criminalidade. Algumas
leis germânicas estatuem que nenhum advogado se poderá subtrair à
obrigação da defesa com o pretexto de nada achar que opor à acusação.
No juramento imposto pela lei genebrina de 11 de julho de 1836,
juramento no qual se compendiam os deveres do advogado, entre outras
promessas, que se lhe exigem, se encontra a de “não aconselhar ou
sustentar causa, que lhe não pareça justa, \textit{a menos que se trate
da defesa de um acusado}”. Ante a justiça primitiva, pois, o patrocínio
de uma causa má, não só é \textit{legítimo, senão ainda obrigatório};
porquanto a humanidade o ordena, a piedade o exige, o costume o
comporta, a lei o impõe (\textit{L’Avvocatura}, pp.~160--161). 
\end{hedraquote}

Na grande obra de Campani sobre a defesa penal se nos depara a mesma lição. 
Nos mais atrozes crimes, diz ele, 

\begin{hedraquote}
por isso mesmo que sobre o indivíduo pesa
a acusação de um horrível delito, expondo"-o a castigos horríveis, é que
mais necessidade tem ele de assistência e defesa (\textit{La Difesa
Penale}, vol.~\textsc{i}, pp.~39--41). 
\end{hedraquote}

O Professor Christian, anotando
os \textit{Comentários} de Blackstone (\textsc{iv}, 356), diz: 

\begin{hedraquote}
Circunstâncias pode haver, que autorizem ou compilam um advogado 
a enjeitar a defesa de um cliente. Mas não se pode conceber uma causa, 
que deva ser rejeitada por quantos exerçam essa profissão; visto como esse
procedimento de todos os advogados tal prevenção excitaria contra a
parte, que viria a importar quase na sua condenação antes do julgamento. 

\textit{Por mais atrozes que sejam as circunstâncias contra
um réu}, ao advogado sempre incumbe o dever de atentar por que o seu
cliente não seja condenado senão de acordo com as regras e formas, cuja
observância a sabedoria legislativa estabeleceu como tutelares da
liberdade e segurança individual. 
\end{hedraquote}

As falhas da própria incompetência
dos juízes, os erros do processo são outras tantas causas de
resistência legal da defesa, pelas quais a honra da nossa profissão tem
o mandato geral de zelar; e, se uma delas assiste ao acusado, cumpre
que, dentre a nossa classe, um ministro da lei se erga, para estender o
seu escudo sobre o prejudicado, ainda que, diz o autor de um livro
magistral sobre estes assuntos, “daí resulte escapar o delinquente”
(William Forsyth. \textit{Hortensius}, pp.~388--389; 408--409). 

Nesse tratado acerca da nossa profissão e seus deveres, escrito com a alta moral e o
profundo bom"-senso das tradições forenses da Grã"-Bretanha, se nos
relata o caso da censura articulada pelo Lord Justice"-Clerk, no
processo de Gerald, réu de sedição, que, em 1794, requeria às justiças
de Edimburgo lhe nomeassem defensor, queixando"-se de lhe haverem
negado os seus serviços todos os advogados, a cuja porta batera.
“Ainda sem a interferência deste tribunal”, admoestou o magistrado, a
quem se dirigia a petição, 

\begin{hedraquote}
nenhum \textit{gentleman} devia recusar"-se a
defender um acusado, \textit{fosse qual fosse a natureza do
seu crime} [\textit{whatever the nature of his crime might be}].
\end{hedraquote}

 De tal modo calou nos ânimos essa advertência, que Howell, o
editor dos \textit{Processos de Estado}, endereçou uma nota ao decano
da Faculdade dos Advogados Henry Erskine, irmão do famoso Lord Erskine,
o Demóstenes do foro inglês, único do seu tempo a quem cedia em
nomeada, e Henry Erskine se apressou em responder que o acusado o não
procurara: 

\begin{hedraquote}
Tivesse ele solicitado o meu auxílio, e eu lhe assistiria
[\ldots{}] pois sempre senti, como o Lord Justice"-Clerk, que se não deve
recusar defesa a um acusado, \textit{qualquer que seja a natureza do}
\textit{seu crime}\textit{; whatever be the nature of his
crime} (William Forsyth. \textit{Hortensius}, p.~388). 
\end{hedraquote}

Do que a esse respeito se usa e pensa nos Estados Unidos, temos documento categórico
no livro escrito sobre a ética forense por um eminente magistrado
americano, o Juiz Sharswood da Suprema Corte da Pensilvânia.
Professando, na universidade desse estado, sobre os deveres da nossa
profissão, ensinava ele aos seus ouvintes: 

\begin{hedraquote}
O advogado não é somente o mandatário da parte, senão também um 
funcionário do tribunal. À parte
assiste o direito de ver a sua causa decidida segundo o direito e a
prova, bem como de que ao espírito dos juízes se exponham todos os
aspectos do assunto, capazes de atuar na questão. Tal o ministério, que
desempenhava o advogado. Ele não é moralmente responsável pelo ato da
parte em manter um pleito injusto, nem pelo erro do tribunal, se este
em erro cair, sendo"-lhe favorável no julgamento. Ao tribunal e ao júri
incumbe pesar ambos os lados da causa; ao advogado, auxiliar o júri e o
tribunal, fazendo o que o seu cliente em pessoa não poderia, por míngua
de saber, experiência ou aptidão. O advogado, pois, que recusa a
assistência profissional, \textit{por considerar, no seu
entendimento, a} \textit{causa como injusta e indefensável, usurpa as
funções, assim do} \textit{juiz, como do júri} (\textit{An Essay on
Professional Ethics}, pp.~83--86). 
\end{hedraquote}

Páginas adiante (89-91) reforça o
autor ainda com outras considerações esta noção correntia, que ainda
por outras autoridades americanas vamos encontrar desenvolvida com
esclarecimentos e fatos interessantes (Henry Hardwicke. \textit{The Art
of Winning Cases}. New York, 1896, p.~457, nº \textsc{xv};
Snyder. \textit{Great Speeches by Great} \textit{Lawyers}. New York,
1892, p.~372). 

Ante a deontologia forense, portanto, não há acusado,
embora o fulmine a mais terrível das acusações, e as provas o
acabrunhem, que incorra no anátema de \textit{indigno de defesa}. “A
humanidade exige que todo o acusado seja defendido”
(Mollot. \textit{Règles de la} \textit{Profession d'Avocat}, t.~\textsc{i}, 
p.~92, \textit{apud } Sergeant.  \textit{De la Nature} \textit{Juridique
du Ministère de L'Avocat}, pp.~74--75). 

Lachaud não recusa assistência da
sua palavra a La Pommérais, ladrão e assassino, que, depois de ter
envenenado friamente a sua sogra, envenena com os mesmos requisitos de
insensibilidade e perfídia a mulher que o amava, para se apoderar do
benefício de um seguro, que, com esse plano, a induzira a instituir em
nome do amante, cuja celerada traição não suspeitava. Já vimos que o
grande orador forense não se dedignou de patrocinar a causa de
Troppmann. Na crônica do crime não há muitos vultos mais truculentos.
De uma assentada; sem ódio, sem agravo, por mera cobiça de ouro, matara
uma família inteira: o casal, um adolescente de 16 anos, quatro
meninos, dos quais o mais velho com treze anos e uma criancinha de
dois. Pois esse monstro teve por defensor o advogado mais em voga do
seu tempo. 

Nunca, desde o processo Lacenaire, houvera um caso, que levasse a 
indignação pública a um tal auge. Quando o criminoso escreveu a
Lachaud, implorando"-lhe que lhe acudisse, esta sua pretensão de
eleger por patrono aquele, a quem então se começava a chamar, por
excelência “o grande advogado”, ainda mais irritou a cólera popular; e,
ao saber"-se que ele aceitara a defesa do matador de crianças, cuja
causa a multidão queria liquidar, linchando o grande criminoso, não
se acreditou, protestou"-se, tentou"-se demovê"-lo, e deu"-se voz de
escândalo contra essa honra a tão vil aborto da espécie humana. 

Mas ao mundo forense essas imprecações e clamores não turvaram a serenidade.

\begin{hedraquote}
O advogado, fosse quem fosse, que Troppmann escolhesse, teria, nestas
tristes circunstâncias, cumprido o seu dever honestamente, como querem
a lei e o regimento da Ordem.
\end{hedraquote}

Lachaud, impassível ao vozear da ira pública, apresentou"-se com simplicidade 
ao tribunal, diz o editor de seus discursos, 

\begin{hedraquote}
como auxiliar da justiça, para ajudá"-la a se
desempenhar dos seu deveres, e, \textit{como defensor, para levantar
entre o culpado e} \textit{os ardores da multidão uma barreira}.
\end{hedraquote}

 A sua oração ali, obra"-prima de eloquência judiciária e consciência
jurídica, abre com estes períodos de oiro: 

\begin{hedraquote}
Troppmann me pediu que o
defendesse: é um dever o que aqui venho cumprir. Poderão tê"-lo visto
com espanto os que ignoram a missão do advogado. Os que dizem haver
crimes tão abomináveis, tão horrendos criminosos que não há, para eles,
a mínima atenuante na aplicação da justiça, os que assim entendem,
senhores, laboram em engano, confundindo, na sua generosa indignação, a
justiça com a cólera e a vingança. Não percebem que, abrasados nessa
paixão ardente e excitados da comiseração para com tantas vítimas,
acabam por querer que se deixe consumar um crime social, de todos o
mais perigoso: o sacrifício da lei. Não compreendo eu assim as
obrigações da defesa. O legislador quis que, ao lado do réu, fosse quem
fosse, houvesse sempre uma palavra leal e honrada, para conter, quanto
ser possa, as comoções da multidão, as quais, tanto mais terríveis
quanto generosas, ameaçam abafar a verdade. A lei é calma, senhores:
não tem jamais nem sequer os arrebatamentos da generosidade. Assentou
ela que a verdade não será possível de achar, senão quando buscada
juntamente pela acusação e pela defesa. Compreendeu que nem tudo está
nas vítimas, e que também é mister deixar cair um olhar sobre o
acusado; que à justiça e ao juiz toca o dever de interrogar o homem,
sua natureza, seus desvarios, sua inteligência, seu estado moral. Ao
advogado então disse: “Estarás à barra do Tribunal, lá estarás com a
tua consciência”. [...] O direito da defesa, a liberdade da defesa,
confiou"-os à honra profissional do advogado, conciliando assim os
legítimos direitos da sociedade com os direitos não menos invioláveis
do acusado.

[\ldots{}]

Houve algum dia, senhores, uma causa criminal, que mais exigisse a
audiência da defesa? Malvadezas sem precedente [...] e no meio desta
emoção geral, clamores exaltados a exigirem, contra o culpado,
severidades implacáveis. Não avaliais, senhores, que a palavra de um
defensor vos deve acautelar desse perigo? Jurastes não sacrificar os
interesses da sociedade, nem os do acusado; prometestes ser calmos,
inquirir da verdade fora das paixões tumultuosas da multidão; jurastes
deixar falar a vossa consciência, quando se recolher, depois de tudo
ouvido. 

Pois bem! eu vo"-lo exoro, impondo silêncio às vossas
consciências, tende essa coragem, e esperai! 
\end{hedraquote}

Onze anos antes os auditórios de Paris se haviam agitado 
aos debates de um processo, que ainda mais comovera a sociedade francesa. 

Um atentado extraordinário
estremecera a nação toda, abalando o mundo político até os
fundamentos. 

O Império escapara de soçobrar num momento, fulminado, nas
pessoas do Imperador e da Imperatriz, pela audácia de um tenebroso
conspirador.

 A mais miraculosa das fortunas salvara do excídio a Napoleão \textsc{iii}, com o
chapéu varado por uma bala e o próprio rosto escoriado. 

Mas os estragos em torno dele operados foram medonhos. 

Dilacerado o carro imperial
pelas estilhas da carga homicida, os animais ficaram vasquejando, num
charco de sangue, de envolta com uns poucos de agonizantes: lanceiros,
gendarmes, lacaios, transeuntes, alcançados todos pela ação
exterminadora das bombas. 

A estatística dessa devastação instantânea
contou 511 ferimentos, 148 feridos e oito mortos. Dificilmente se
poderia improvisar de um só golpe maior número de infortúnios e
sofrimentos. O fulminato de mercúrio obrara maravilhas de
instantaneidade na supressão de vidas inocentes; e a influência maligna
dos projetis empregados revestira um caráter singularmente desumano,
condenando os sobreviventes, pela natureza das chagas abertas nos
tecidos lacerados, a cruciadores tormentos, ou moléstias incuráveis.

Tal se apresentara a obra da sanguinária conjura, que imortalizou com
uma auréola negra o nome de Felice Orsini. 

As intenções, que a haviam
animado, não menos sinistras. “Pouco importava”, diz o historiador do
Segundo Império, 

\begin{hedraquote}
que os estilhaços, projetando"-se por toda a parte,
juntassem à grande vítima votada à morte um sem conto de vítimas
obscuras. Pouco importava, contanto que se imolasse o Imperador.
Reinaria então a anarquia em França, mediante a sua repercussão a
anarquia na Itália, e destarte, se realizariam os pavorosos sonhos
dessas imaginações doentias e pervertidas (De la Gorce, \textsc{ii}, 219).
\end{hedraquote}

Pois bem: a esse crime, de tão infernal aspecto e tão bárbaras entranhas,
não faltou, no julgamento sem conforto de esperança, a mão piedosa de
um advogado, e esse o maior dos contemporâneos, aquele que exercia
então sobre a sua classe o principado da eloquência e da celebridade
profissional. 

Todos se inclinaram com admiração e respeito a esse ato
de religiosa solenidade. Ninguém tolheu a defensiva ao execrado réu,
cuja altivez de recriminações levou o primeiro presidente do tribunal a
declarar"-lhe que só o respeito às liberdades da defesa o obrigara a
tolerar semelhante linguagem; e foi sobre a cabeça do réprobo, 
escoltado de espectros, que a inspirada oração de Júlio Favre ousou
acabar, apelando das durezas da justiça da terra para as equidades da
clemência do céu. “Para cumprirdes o vosso dever sem paixão nem
fraqueza”, dizia ele em acentos de Bousset, 

\begin{hedraquote}
não haveis mister,
senhores, as adjurações do Sr.~Procurador"-Geral. Mas Deus, que a todos
nos há de julgar; Deus, ante quem os grandes deste mundo comparecem
tais quais são, despojados do séquito dos seus cortesãos e lisonjeiros;
Deus que mede, ele só, a extensão das nossas culpas, a força dos
impulsos que nos desvairam, a expiação que os resgata; Deus
pronunciará, depois de vós, a sua sentença: e talvez não recuse o
perdão, que os homens houverem tido por impossível na terra. 
\end{hedraquote}

Bem vê, pois, o meu colega: não há de que se arrepender. Tem consigo a lição
geral e os melhores exemplos da nossa gloriosa profissão. 

Há de lhe ser árdua a tarefa. Não vejo na face do crime, cujo autor vai defender,
um traço, que destoe da sua repugnante expressão, que lhe desbaste o
tipo da refinada maldade. 

Fala"-me em elementos, de que está de posse,
os quais “muito diminuem, se não excluem, sua responsabilidade”. Queira
Deus que se não iluda. Essa responsabilidade se acentua, no conjunto
das provas conhecidas, com uma evidência e uma proeminência, que se me
afiguram insusceptíveis de atenuação. 

Nem por isso, todavia, a assistência do advogado, na espécie, 
é de menos necessidade, ou o seu papel menos nobre. 
\bigskip

\hfill\textit{Rui Barbosa} 

\clearpage
