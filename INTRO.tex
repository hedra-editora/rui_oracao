\chapter[Introdução, \emph{por Pedro Luso}]{Introdução}
\hedramarkboth{introdução}{Pedro Luso}

\begin{flushright}
\textsc{pedro luso}
\end{flushright}

\noindent{}Os formandos da turma de 1920, da Faculdade de Direito de São Paulo,
escolhem para seu paraninfo Rui Barbosa, que, por encontrar"-se enfermo,
não pode comparecer à solenidade. Sua ausência é sentida
pelos formandos, professores e convidados, que se encontram no recinto.
Quem não gostaria de ouvir o grande orador assomar à tribuna para
saudar os jovens afilhados, e do mestre ouvir palavras sábias,
conselhos importantes, para quem deixa os bancos acadêmicos, e, daí em
diante, passa a enfrentar a crueza da vida profissional, na área do
Direito? Mas Rui já havia escrito o seu discurso. No momento mais
solene da formatura, a saudação aos futuros bacharéis, as ideias de Rui
iam sendo absorvidas por eles, à medida em que o professor Reinaldo
Porchat, que o substituía na tribuna, lia o discurso escrito para a
ocasião, discurso esse, que passaria a ser conhecido como \textit{Oração aos
moços}, e reconhecido como a obra"-prima de Rui Barbosa. 

Ali está registrado, no introito do
famoso discurso, que é sacerdócio o exercício do Direito. E, ainda,
nessa sua primeira parte, Rui dá realce ao contraste existente entre
paraninfo e paraninfados: o primeiro, alquebrado pela passagem dos
anos, embora guardando bem nítida na memória o muito que fez ao longo
dos seus cinquenta anos de exercício da profissão de advogado; os
segundos, jovens, que se iniciam como juízes e advogados. Diz aos
formandos, não admitir que o despreparo do jovem juiz possa justificar
atos, por eles praticados, contrários às leis e à dignidade humana.
Deste ponto em diante, Rui passa a ministrar
sábios conselhos a seus afilhados, com destaque aos deveres que os
futuros juízes e advogados terão de observar. 

O juiz, certamente encontrará riscos no desempenho de suas funções, diz
Rui, mas terá de ter coragem para enfrentá"-los. Essa é uma sagrada
exigência da sociedade; é a exigência de cada cidadão, que, a qualquer
preço, quer ver resguardado o direito que lhe outorga a Constituição.
Portanto, ao juiz não cabe pretextar a má aplicação da lei, por
desconhecê"-la na sua inteireza. Conhecer a lei é dever intrínseco ao
cargo do juiz. Para Rui, em outras palavras, o juiz que protela o
julgamento de uma ação, comete crime contra a parte que está sob o
manto da lei. Um julgamento demorado pode significar prejuízo, mesmo
para quem sai vencedor da demanda, em razão do atraso para ser julgado
o feito; pelos anos ou mesmo décadas, que os autos ficam estagnados na
mesa do juiz. ``Mas justiça atrasada'', diz Rui, ``não é justiça, senão
injustiça qualificada e manifesta''. No seu discurso, o paraninfo pede
aos formandos, que vão escolher como profissão a magistratura, que não
se tornem magistrados como esses, que deixam os autos penar em suas
mãos como as almas do purgatório. Insta a seus afilhados, futuros
juízes, a honrar a justiça. Insta"-os, também, a se entregarem ao
trabalho com a dignidade que o cargo requer. 

O paraninfo lembra aos formandos --- não exatamente com essas palavras ---,
que, na militância da advocacia, o advogado verá que se desenvolve
nessa sua missão, um pouco do sentimento de magistrado. O mestre, mais
uma vez, está com a razão. E, esse entrelaçamento, que se desenvolve no
exercício da advocacia, sente"-o o advogado em muitos casos ou fases
processuais, como no exemplo que segue: o advogado é procurado para
defender direito supostamente lesado; o consulente narra os fatos
relacionados com a lesão, e exibe"-lhe os documentos necessários à
produção de provas, para acompanhar o pedido pretendido. Poder"-se"-ia
perguntar se, de pronto, o advogado fica convencido de que tal direito
fora efetivamente lesado. É justamente neste ponto que se encaixa a
preleção de Rui, no respeitante à espécie de magistratura, que 
desenvolve"-se na missão do advogado: aí, em cumprimento a um dever
profissional, o advogado terá de avaliar criteriosamente fatos e
documentos, para sentir se há probabilidade de êxito no pleito, caso a
ação venha a ser ajuizada com o seu patrocínio; e, assim procedendo,
terá sido ele, o advogado, o primeiro juiz da causa. 

Fala ainda sobre a vocação do advogado
e sobre a síntese de todos os seus mandamentos: legalidade e liberdade.
Para Rui, o advogado por vocação não deserta e nem corteja a justiça, e
não deixa de ser"-lhe fiel. Insiste Rui que o advogado vocacionado não
passa da legalidade para a violência, e mantem"-se nos limites da ordem,
sem atender os apelos da anarquia. Por todo o elenco de conselhos
dirigidos ao advogado, a leitura desse discurso será de
grande valia, tanto para o estudante, como ao advogado. 
É obra que poderá forjar o caráter do jovem
estudante, ajudando"-o a preparar"-se para os embates da profissão, sem
declinar da ética, que deverá pautar a sua caminhada, da qual poderá
orgulhar"-se ao longo dessa trajetória. Conduta que poderá constituir"-se
em orgulho para si próprio, para a família, e, mais acima ainda,
orgulho para a pátria.

Interessante observar que na fala de Rui em \textit{Oração aos moços} está
bem forte a presença de Deus, cuja ênfase não era dada pelo jurista nos
seus tempos áureos, em que o corpo não denunciava sinal de fragilidade,
e sua atenção estava voltada para os embates sucessivos no Senado, nos
quais enfrentou bravamente ferrenhos adversários; nesses tempos, reserva
suas forças para atuar no terreno inóspito da política; pretende levar
a termo sua missão contra o opróbrio da escravidão, que aspira
colocar"-lhe cobro; guarda"-as, também, para o enfrentamento contra os
monarquistas, que quer derrotá"-los, e impor"-lhes a república, como
forma de governo, mais consentânea com o seu tempo e com as aspirações
dos brasileiros. Agora, no entanto, nos poucos anos que lhe restam,
busca a harmonia em Deus. Rui acredita na oração e no trabalho. A
oração é, para ele, o meio que forma a moral do homem. O trabalho é o
meio que reúne energia do corpo e do espírito.

Com \textit{Oração aos moços}, Rui Barbosa atinge o seu pináculo como escritor.
As lições que transbordam dessa requintada obra não existiriam se quem
a escreveu não tivesse a sensibilidade, a cultura, a sabedoria e a
honestidade de Rui. É pois, como que um atestado,
passado pelo escritor, do conhecimento que tão árdua e severamente
buscou no decorrer de sua vida. Sabia Rui que a sua inteligência não
seria o suficiente para galgar elevados postos, como aspirava. Para
isso, teria que submeter"-se a rígido programa de estudo, como sempre
fez no seu longo aprendizado, aprendizado esse que nunca chegou a
concluir, e que jamais concluiria, porque era de seu temperamento a
busca da perfeição; sabia, isto sim, que muito ainda teria por
aprender, e, principalmente, por muito que meditar, sobre o que lhe
diziam, em suas obras, os mestres do Direito, da Filosofia, da
Literatura, da Pedagogia e de outras ciências, com as quais estava
familiarizado. Obra escrita em homenagem aos
formandos que o convidaram como paraninfo, Rui pretendeu fazer um
resumo de toda a sua vivência para transmiti"-la a esses jovens, como
que pretendendo deixar"-lhes um legado de sua genialidade e de seu
exemplo de honradez. Rui deixa exemplo imperecível de obstinação e de
trabalho. Para Rui, a liberdade é o seu bem maior. 

Pelo discurso proferido na Biblioteca Nacional, no ano de 1918,
portanto, dois anos antes da solenidade de formatura dos seus
afilhados, tem"-se o privilégio de saber o que o autor de \textit{Oração aos
moços} diz de si mesmo: 

\begin{hedraquote}
Propugnei ou adversei governos; golpeei
governos; golpeei ou escudei instituições; abalei até à morte um
regímen, e colaborei decisiva e capitalmente no erigir de outro.
Pelejei contra ministros e governos, contra prepotências e abusos,
contra oligarquias e tiranos. Ensinei, com a doutrina e o exemplo, mas
ainda mais com o exemplo que com a doutrina, o culto e a prática da
legalidade, as normas e o uso da resistência constitucional, o desprezo
e o horror da opressão, o valor e a eficiência da justiça, o amor e o
exercício da liberdade.\footnote{ Discurso na Biblioteca Nacional, por ocasião
do jubileu cívico, em 1918.  Gladstone Chaves de Melo. \textit{Rui 
Barbosa. Textos Escolhidos.} 2ª ed. Rio de Janeiro: Livraria Agir Editora, 
1968, p.~7.}
\end{hedraquote}


Para compreendermos a sua carreira, faz"-se necessário apresentar alguns traços
de sua biografia.  Rui Caetano Barbosa de Oliveira, que viria
consagrar"-se com o nome de Rui Barbosa, era filho de Dr.~João José
Barbosa de Oliveira, médico, e de D.~Maria Adélia Barbosa de Oliveira, e nasceu 
na cidade de Salvador, Bahia, no dia 5 de novembro de 1839. Quando faleceu em 
Petrópolis, em 10 de março de 1923, aos 73 anos, Rui Barbosa ainda era senador. 

Rui Barbosa iniciou seus estudos primários com o professor Antônio Gentil
Ibirapitanga, em 1854. Em 1861, inicia os estudos secundários no
Ginásio Baiano, de Abílio César Borges. Em 1866, ingressa no curso de
Direito da Faculdade de Recife, de onde se transfere para a capital
paulista. No ano de 1868, matricula"-se na Faculdade de Direito de São
Paulo, onde teve como colegas Castro Alves e Joaquim Nabuco. Em 28 de
outubro de 1870, dá"-se a sua colação de grau de bacharel em Direito. Em
1871, inicia sua carreira de advogado, com Sousa Dantas e Leão Veloso.
Em 1872, começa a escrever para o Diário da Bahia, de Manuel Dantas. Em
1876, fica noivo de Maria Augusta Viana Bandeira, na Bahia. Nesse mesmo
ano, vai tentar a fortuna no Rio. Em dezembro, volta à Bahia para a 
cerimônia de casamento com Maria Augusta. Nos anos que se seguem, 
nascem os seus cinco filhos: Maria Adélia, Francisca, Maria Luísa, Alfredo Rui e João Rui.

Regressa à Bahia em 1877, para tentar a carreira política. O exercício
da advocacia não lhe basta, a política é igualmente sua aspiração. Para
esse mister, nada melhor que começar na província. 
O tempo em que fica no Rio serve para que se muna de forças
para decidir sobre o futuro que almeja. Agora, está pronto para os
grandes debates. E, para que possa dedicar"-se às atividades políticas,
exonera"-se do emprego na Santa Casa da Misericórdia, abre escritório de
advocacia e passa a colaborar de forma sistemática no Diário da Bahia.
Assim, Rui vai construindo a base para tentar, mais tarde, galgar
cargos eletivos.

Nessa época, Rui passa por enormes dificuldades, e a Bahia não se
apresenta como a melhor solução de vida para ele, com salário
insuficiente para atender a todos os seus compromissos, dentre eles o
pagamento de muitas dívidas contraídas pelo seu pai, que falece de
forma inesperada. Conta esse seu sofrimento em carta: 

\begin{hedraquote}
A minha vida é um labor e agonia muito acima de minhas forças, não
morais, mas físicas. O meu emprego na Misericórdia, a advocacia, aqui
trabalhosíssima, bem quase estéril, e a imprensa política, de que me é
impossível separar, constituem três séries de deveres, cada uma das
quais seria suficiente para ocupar um homem são, e que, entretanto,
vejo"-me forçado a reunir em mim só. Demais, a inopinada morte de meu
pai deixou"-me pai de família, e sobrecarregado com uma dívida superior
a doze contos de réis, quase todas em letras bancárias, além de outras
obrigações pecuniárias muito graves. Tenho letras em todo os meses do
ano, doze em cada dentre oito, e três nos restantes. Os meus
vencimentos da Santa Casa (R.s200\$000) não me chegam nem para a
amortização desses débitos, quanto mais para o mais, que excede a outro
tanto daquela quantia. A advocacia nesta província, mendiga, e dia para
dia desce desanimadoramente. Vejo"-me, portanto, continuamente nos
maiores apuros; e, se deles me tenho saído, sabe Deus à custa de quanta
seiva de uma vida que, neste ano só, se tem esgotado como dez.\footnote{ \textit{In}: 
Francisco de Assis Barbosa. \textit{Retratos de Família}. 2ª ed. Rio de 
Janeiro: Livraria José Olympio Editora, 1968, p.~30.}
\end{hedraquote}

Em 1878, Rui é eleito deputado provincial pelo Partido Liberal. Nesse
mesmo ano do seu retorno à Bahia, surge para Rui a grande oportunidade,
quando o general Osório visita Salvador, ocasião em que é levado pelo
povo até o Diário da Bahia; aí, aparece numa das janelas o jovem Rui
Barbosa, representante do Partido Liberal, para saudar o herói da
Guerra do Paraguai. Rui faz um panegírico a Osório, que é senador pelo
Partido Liberal, enquanto Caxias é por ele ironizado de forma
contundente. Com isso, o caminho fica aberto. Então, Rui é eleito à
Assembleia Provincial da Bahia. Depois dessa fase inicial, é eleito
para Câmara, instalada em 1879. Elege"-se Senador pela Bahia, em 1890,
mandato que manteria até o final de sua vida. Por duas vezes,
candidata"-se à Presidência da República, e, em ambas as candidaturas, é
derrotado; a primeira, para o Marechal Hermes (1909), e a segunda, para
Epitácio Pessoa (1919).

Nessa época (1919), são eleitos para a Câmara os moços Joaquim Nabuco,
Afonso Pena, José Mariano, Buarque de Macedo. Também são eleitos
Rodolfo Dantas e Rui Barbosa. Entre os mais velhos estão: o republicano
Saldanha Marinho, o notável jurista Lafayette Pereira, e Silveira
Martins, um dos mais brilhantes oradores do segundo reinado. É nesse
ambiente que a figura de Rui começa a firmar"-se com as suas intervenções
nos grandes debates. No ano de 1881, Rui, agora deputado geral, embarca
no porto da Bahia, com a mulher e as três filhas, com destino ao Rio de
Janeiro, onde o êxito absoluto o aguardava. Um dos mais importantes
objetivos de Rui, agora no exercício de cargo político, é empregar
todas as armas que tem à sua disposição: a eloquência e a sua escrita 
brilhante --- armas essas já
conhecidas por muitos na época ---, com o objetivo de ver a justiça
triunfar. O homem culto, inteligente e com incomum disposição para o
trabalho, dedica quase toda sua vida, a contar de seus dezoito anos até
a idade de 73 anos, às causas nacionais.

Rui Barbosa inicia no jornalismo na sua cidade natal, Salvador, no
\textit{Diário da Bahia}, e continuou no Rio de Janeiro, em \textit{O País} e no \textit{Jornal
do Comércio}. Eleito deputado a Assembleia Provincial baiana (1878), no
ano seguinte deputado"-geral, distingue"-se na elaboração da reforma
eleitoral e em pareceres sobre o ensino. Desliga"-se do Partido Liberal,
durante o último gabinete da monarquia, e passa a defender o regime
federativo. Desfruta o prestígio de ser vice"-presidente do governo
provisório e Ministro da Fazenda. Redige a primeira Constituição
republicana, em 1891. Toma parte da Revolta Armada, em 1893, e, em
razão dessa posição que toma, exila"-se em Buenos Aires e depois em
Londres; na capital inglesa, permanece por dois anos. De volta ao
Brasil, por duas vezes, candidata"-se à presidência da República, em
1910 e 1919, respectivamente. Não consegue os votos necessários para
ser guindado ao cargo supremo da nação. 

Rui Barbosa participa da votação do Código Civil. O projeto de lei desse diploma
legal coube a Clóvis Beviláqua. Rui, no entanto, acrescenta ao projeto numerosas
emendas de ordem gramatical e filológica.  Em razão dessas interferências,
origina-se a célebre polêmica entre Rui e Ernesto Carneiro Ribeiro (seu antigo
mestre na Bahia), então o responsável pela revisão do Código, que culminou em um
dos textos ruianos mais famosos: \textit{A réplica}, lançada em 31 de dezembro de 1902.
Na época, essa polêmica toma grandes proporções, envolvendo Rui Barbosa e
Carneiro Ribeiro, duas respeitáveis figuras, admiradas pela erudição e pela
honradez. Rui, que se consagra como profundo conhecedor dos clássicos do idioma,
passa a integrar a Corte Permanente Internacional da Justiça.

Em 1910, Rui lidera o primeiro movimento significativo de opinião
pública, na célebre Campanha Civilista. Nos anos de 1906 a 1909, foi,
respectivamente, senador pela Bahia e vice"-presidente do Senado. Em
1907, representa o Brasil na Segunda Conferência de Paz, em Haia,
Holanda, alcunhado desde então a ``Águia de Haia''.

Rui Barbosa foi membro fundador da Academia Brasileira de Letras.
Escreveu, entre outras importantes obras: \textit{Queda do Império} (1889);
\textit{Cartas de Inglaterra} (1896); \textit{Parecer sobre a redação do Código Civil}
(1902); \textit{Réplica às defesas de redação do projeto da câmara dos
deputados} (1903); \textit{Oração aos moços} (1920).

O autor também dá importante contribuição à pedagogia. É o precursor
da pedagogia no Brasil, o primeiro pedagogo que se dedica ao problema
integral da cultura, isto é, o problema filosófico, social, político e
técnico, a um só tempo. A sua obra, ligada à pedagogia, é amplamente
reconhecida, como se vê pela repercussão, do que preleciona nessa
área, em obras de autores famosos, como Veríssimo, Romero, Bordeaux
Rêgo, Monteiro de Sousa, José Augusto, Sampaio Dória, Carneiro Leão,
Afrânio Peixoto, Miguel Couto, Teixeira de Freitas, Mário Pinto Serva,
entre outros. 

Cumpre lembrar que Rui produziu uma alentada obra, sete a oito tomos, em
prol da pedagogia no Brasil, como menciona Lourenço Filho, cujos
escritos pedagógicos integram a coleção das suas Obras Completas.
Portanto, quaisquer que tenham sido as circunstâncias, a forma de
produção e a extensão, bastariam esses trabalhos para que o nome do
autor fosse incluído no rol de nossos maiores pedagogistas, e para que
o seu pensamento tivesse influído, como continua a influir, sobre
gerações sucessivas de mestres e estudiosos da especialidade, em nossa
terra. Tais escritos, contudo, não se separam do conjunto da vida
pública de Rui, nela representando, por vezes, a chave da compreensão
de muitas passagens de suas lutas, e de mudanças que, em certas ideias
e atitudes, apresentou.

\begin{hedraquote}
Se os pedagogos são fastidiosos, o mesmo não
se poderá dizer da pedagogia, quando haja a oportunidade de
contemplá"-la em suas legítimas dimensões. Isto é, desde que com ela
tomemos contato em concepções dotadas de grandeza de linhas, força de
estrutura e maior sentido de compreensão humana. A pedagogia de Rui
Barbosa reveste"-se desses admiráveis atributos, muito embora não haja
sido ele educador de ofício, ou, talvez por isso mesmo. Rui não figura
como profissional do ensino. Salvo pequena participação que deu a um
curso noturno para analfabetos, quando estudante em São Paulo, não
exerceu o magistério; também não desempenhou cargos de administração,
não foi inspetor de ensino ou diretor de escola. Seus escritos sobre
educação, todos produzidos no limitado prazo de um lustro, precisamente
o que mediou de 1881 a 1886, tiveram caráter episódico, decorreram da
vida política, foram aspectos da luta do doutrinador e reformador
social. Nessa época, andava entre os 32 e 37 anos de idade e era
deputado pela Bahia.\footnote{M. B. Lourenço Filho. \textit{A pedagogia de Rui Barbosa}. 
3ª ed. São Paulo: Edições Melhoramentos, 1966, p.~11.}
\end{hedraquote}

Diz o autor, que a produção pedagógica de Rui como que delimita, no
tempo, com a publicação do livro \textit{Lição de coisas}, do educador
norte"-americano Norman Allyson Calkins, traduzido em 1881, e somente
sendo impresso em 1886. Afirma ainda, que depois da publicação dessa
obra Rui não voltou a ocupar"-se de temas de ensino excetuados artigos
de imprensa, não muitos, e que por sua natureza mais judiciosamente se
hão de classificar na produção jornalística.

Ainda sobre a produção pedagógica de Rui, diz Lourenço Filho: 

\begin{hedraquote}
Como se vê, no acervo imenso de cultura que nos legou, os escritos pedagógicos
representam parcela relativamente diminuta. Mas, em Rui, o diminuto é,
ainda e sempre, copioso. Na coleção das Obras Completas os escritos
pedagógicos darão de sete a oito tomos. Quaisquer que tenha sido,
aliás, as circunstâncias, a forma de produção e a extensão, bastariam
esses trabalhos para que o nome do autor fosse incluído no rol de
nossos maiores pedagogistas, e que para que o seu pensamento tivesse
influído, como continua a influir, sobre gerações sucessivas de mestres
e estudiosos da especialidade, em nossa terra.\footnote{\textit{Id.~Ibid.}~p.~12.} 
\end{hedraquote}

O professor Lourenço Filho escreve, a seguir, sobre as várias razões da
importância da atuação de Rui como pedagogo: 

\begin{hedraquote}
A primeira está em que
Rui, como em tanta outra coisa, aí figura como precursor. Foi sem
dúvida, no Brasil, o primeiro a tratar da pedagogia como problema
integral da cultura, isto é, problema filosófico, social, político e
técnico, a um só tempo. A segunda é que tais escritos não se separam do
conjunto de sua vida pública, nela representando por vezes, a chave da
compreensão de muitas passagens de suas lutas, e de mudanças que em
certas ideias e atitudes apresentou. Por último, a oportunidade com que
trabalhou tais assuntos, em momentos de rápida evolução de doutrinas
sociais e educativas no mundo, e em nosso país, 
em particular.\footnote{\textit{Id.~Ibid.}~p.~12.} 
\end{hedraquote}

Atento ao peso da obra pedagógica, Lourenço Filho mostra a repercussão
do trabalho de Rui, dizendo: “Sobre o valor extrínseco ou ostensivo da
obra pedagógica, tal qual o podemos sentir em seus efeitos (e bastará
ver a repercussão de suas páginas em Veríssimo, Romero, Bordeaux Rêgo,
Monteiro de Sousa, José Augusto, Sampaio Dória, Carneiro Leão, Afrânio
Peixoto, Miguel Couto, Teixeira de Freitas, Mário Pinto Serva, para não
citar outros) convirá ressaltar, assim, a importância que apresenta na
exegese da produção total do autor (Rui), ou seja, na posição e
evolução de suas próprias ideias, tendências e sentimentos.”

Segue Lourenço Filho discorrendo sobre o valor extrínseco da obra
pedagógica de Rui, referindo"-se o que representou o discurso de 6 de
maio de 1882, ``com que defendeu o programa não expressamente formulado
do gabinete Martinho Campos; os pequeninos apartes com que pontuou a
justificação de dois projetos de Rodolfo Dantas, sobre a criação de um
liceu feminino e de um fundo escolar nacional; o sensacional discurso
ao ensejo do centenário de Marques de Pombal --- sensacional pela
extensão de quase três horas e pela violência do ataque à Companhia de
Jesus --- quanto estranho, verdadeiramente estranho, pela forma vibrante
com que exalta a figura o autoritário ministro de D.~José, na qual
chega a admitir a liberdade e a igualdade social sem existência de
liberdade política. Será ainda com essa perspectiva que melhor podemos
apreciar o discurso da Volta à Terra Natal; os trechos da introdução de
O Papa e o Concílio, que reedita e comenta na derradeira das Cartas da
Inglaterra aí mesmo, todo o primoroso estudo sobre As Bases da Fé, de
Balfour, os subentendidos da alocução do Colégio Anchieta, a qual
marcou época; e, enfim, os tons de suave melancolia na Oração aos moços
--- isso, já agora em 1920, nas luzes do ocaso. Uma conclusão, portanto,
a retirar desta primeira notícia: quem desejar conhecer Rui há de
conhecer"-lhe a obra pedagógica e meditar nela.''

Francisco de Assis Barbosa, advogado, escritor, professor e jornalista,
publicou, em 2ª ed. pela editora José Olympio, de \textit{Retratos de família},
prefaciada por Josué Montello, na qual enfoca passagens importantes da
vida de ilustres escritores, entre eles, Rui Barbosa. Diz esse premiado
escritor paulista, que Rui Barbosa talvez seja o homem mais admirado no
Brasil. Também enaltece Rui, dizendo não haver ninguém tão culto, tão
sábio, tão honesto, tão perfeito como ele. Realça o sentimento à
liberdade, que, em última análise, é uma constante na vida do jurista
baiano. Há, nessa obra de Francisco Assis Barbosa, uma passagem, da
qual não podemos nos furtar: “Enquanto houver um pária da Justiça, um
perseguido, um vítima da prepotência'', escreveu o poeta Augusto Schmidt,
``ele, Rui, estará sempre presente entre nós''.\footnote{ Francisco de 
Assis Barbosa. \textit{Retratos de família}. 2ª ed. Rio de Janeiro: 
Livraria José Olympio Editora, 1968, p.~36.}

Para Assis Barbosa, a nossa geração sente o peso da sombra de Rui
Barbosa. O seu nome faz lembrar os \textit{meetings} da campanha civilista, os
habeas corpus em favor dos prisioneiros políticos, dos jornalistas
ameaçados, dos oprimidos. Sua voz sempre foi a favor do mais
fraco, uma voz a serviço do povo. Rui é o símbolo de uma coisa que
todavia não está morta: a opinião pública. Ele foi o homem que encarnou
a oposição, o direito da livre crítica contra os atos dos governantes,
o direito de ser contra os poderosos do momento. E com coragem o fazia:

\begin{hedraquote}
Não nasci cortesão. Não fui do trono; não o quis ser da ditadura; 
da própria Nação não o sou: não o
serei das baionetas. Aceitarei a luta no terreno onde a puserem dentro
das leis normais e políticas do dever e da honra. Não está na minha
índole fugir. O meu temperamento, mercê de Deus, não é o dos poltrões.
Da firmeza do meu posto cívico não permitirá Deus jamais me abalem
ameaças ou perigo.\footnote{ In: Francisco de Assis Barbosa. 
\textit{Retratos de família}. 2ª ed. Rio de Janeiro: Livraria José Olympio 
Editora, 1968, p.~35.}
\end{hedraquote}
 
É ainda na obra \textit{Retratos de família}, de Francisco Assis Barbosa, que
encontramos a transcrição de mais um pequeno trecho de um texto escrito
por Rui Barbosa, texto esse que certamente sempre terá espaço e
especial e de destaque, onde quer que lá esteja, à disposição do leitor: 

\begin{hedraquote}
Os governos arbitrários não se acomodam com a autonomia da
toga, nem com a independência dos juristas, porque esses governos vivem
rasteiramente da mediocridade, da desonra. A palavra os aborrece,
porque a palavra é o instrumento irresistível da conquista da
liberdade. Deixai"-a livre, onde quer que seja, e o despotismo está morto.
\end{hedraquote}

O legado da obra de Rui Barbosa, bem como a sua incansável atuação
contra a opressão, sua luta em prol da liberdade, o Rui cerebral,
refletido e consciente artista da palavra, o Rui das escritas buriladas
e requintadas, o Rui dos discursos e das conferências brilhantes e
famosas, o Rui de Oração aos moços, esteve esquecido no período que
compreende os anos de 1930 a 1960. Dois motivos podem ser apontados
como causa desse esquecimento: o modernismo literário e a ditadura 
implantada em 1937 na esteira da revolução de 1930.

Ronald de Carvalho aborda oradores e publicistas na sua importante obra,
\textit{Pequena História da Literatura Brasileira}, 9ª edição revista, publicada
pela F.~Briguiet Editores, Rio de Janeiro, 1953, mencionando, como
digno de nota, entre outros, José Bonifácio de Andrada e Silva, José do
Patrocínio, Joaquim Nabuco e Rui Barbosa. Diz, Ronald de Carvalho: “Rui
Barbosa, cujo estilo é dos mais apurados e elegantes, não só pela
correção da linguagem senão também pela formosura das imagens e dos
tropos”. E, mais adiante: 

\begin{hedraquote}
A campanha abolicionista,
sobretudo, de onde irradiaria a futura propaganda republicana, abriu um
vasto caminho ao conhecimento das ciências políticas e sociais em nosso
país. Oradores como Joaquim Nabuco, poetas como Castro Alves,
polemistas como Tobias Barreto, jornalistas como Quintino Bocaiuva,
publicistas como Rui Barbosa, Salvador de Mendonça e André Rebouças
vieram, com a palavra e a pena, sacudir a perigosa tranquilidade a que
se fizera a nação, imprimindo"-lhe uma certa inquietação necessária ao
bom entendimento dos problemas éticos, religiosos e morais que, então,
pela primeira vez, apareciam aqui em toda a sua plenitude.\footnote{Ronald de 
Carvalho. \textit{Pequena história da literatura brasileira}. 9ª ed. 
Rio de Janeiro: F. Briguiet Editores, 1953, p.~51; 320.}
\end{hedraquote}

\textit{Oração aos moços} é de valor inestimável para a literatura
do nosso país. Nela, a antepenúltima obra escrita por esse exímio
artista da oratória e da escrita, a mais importante escrita pelo
mestre, por certo inspirou Afrânio Coutinho, membro da Academia 
Brasileira de Letras, a escrever: 

\begin{hedraquote}
Ao seu epitáfio, que ele quisera simples e digno de sua
grandeza: amou a justiça, viveu no trabalho e não perdeu o ideal, a
posteridade ajuntará: libertador de cativos, defensor de oprimidos,
educador do povo, reformador da pátria, apóstolo de todas as causas
liberais, o maior dentre os seus no seu tempo, e que soube falar, para
além dos tempos, a eterna linguagem da perfeição literária, mestre
incomparável do verbo.\footnote{ Márcio Tavares d'Amaral. \textit{Rui Barbosa}. 
São Paulo: Editora Três, 1974, p.~252.}
\end{hedraquote}

%
%\chapter[Nota sobre a organização, \emph{por Marcelo Módolo}]{Nota sobre a organização}
%\hedramarkboth{nota sobre a organização}{Marcelo Módolo}
%\vskip-1cm
%
%%\epigraph{\textit{Oração aos moços} é o canto do cisne de Rui Barbosa, 
%%é a mais realizada de suas obras, a que com maior autenticidade, creio, nos dá 
%%a medida e o tom do seu estilo. Disse eu alhures que é ela a obra mais 
%%trabalhada da língua portuguesa.}{Gladstone Chaves de Melo}
%
%\textsc{A última edição} da \textit{Oração aos moços}, 
%devidamente revisada pelo autor, foi a célebre versão
%publicada em 1921 pelo Mensario Academico Dionysos. Segundo Edgard
%Batista Pereira: 
%
%\begin{hedraquote}
%Dionysos foi um mensário acadêmico exclusivamente dedicado às belas 
%letras, num tempo em que os estudantes cuidavam mais de literatura 
%que de direito e mais de direito que de política.
%\end{hedraquote}
%
%Esta edição, que agora publicamos, baseia"-se fundamentalmente na edição
%\textit{Dionysos}, considerada pelos especialistas a edição
%definitiva desse texto ruiano. Com esse texto"-base, foram cotejados os
%seguintes testemunhos: 
%
%\begin{enumerate}%[a.]
%
%\item \textit{Oração aos moços}. Edição
%comemorativa do centenário de nascimento do grande brasileiro.
%Fac"-símile do texto original, datilografado e contendo emendas do
%próprio punho do Autor (São Paulo: Reitoria da Universidade de São Paulo, 1949); 
%
%\item \textit{Oração aos moços}. Estabelecimento do texto,
%prefácio e breves notas explicativas por Carlos Henrique da Rocha Lima
%(Rio de Janeiro: Casa de Rui Barbosa, 1949); 
%
%\item \textit{Oração aos moços}. Estabelecimento do texto e notas 
%de Adriano da Gama Kury. Prefácio de Edgard Batista Pereira 
%(Rio de Janeiro: Casa de Rui Barbosa, 1956) e 
%
%\item \textit{Oração aos moços}. Edição popular anotada
%por Adriano da Gama Kury. 5ª ed. (Rio de Janeiro: Casa de Rui Barbosa, 1999).
%
%\end{enumerate}
%
%
%Para nossa edição, respeitamos a pontuação do texto ruiano, em especial
%o emprego das vírgulas, que sempre tiveram um “quê” de retórico na
%prosa de Rui Barbosa. Assim, leitores menos avisados poderão estranhar
%algumas opções de pontuação, mas que seguem, como já frisamos, a edição
%\textit{Dionysos}. Como algumas tentativas de alterar esse padrão de
%pontuação produziram mais desacertos do que acertos, optamos por manter
%o padrão por ele apresentado. De resto, atualizamos a ortografia, que
%compõe, notadamente, o aspecto mais superficial da língua escrita. 
%
%Em relação às notas, preferimos “pecar pelo excesso que pela falta”,
%mas esperamos não termos cerceado nosso leitor. Com elas,
%contemplamos aspectos linguísticos, literários e históricos, além de
%apontarmos algumas preciosidades desse texto, mas ainda longe de esgotá"-las.
%\ \\
%
%\hfill\textit{Marcelo Módolo}

%\pagestyle{empty}
\begin{bibliohedra}

\vspace*{2ex}

\item \scriptsize\textbf{Obras linguísticas e gramaticais:}

\tit{DIAS}, Augusto Epiphanio da Silva. \textit{Syntaxe historica portuguesa}.
Lisboa: Livraria Clássica Editora de A. M. Teixeira, 1918.

\tit{Grande enciclopédia portuguesa e brasileira}. Vol.~\textsc{v}, Editorial
Enciclopédia: Lisboa/ Rio de Janeiro (s.d.)

\tit{HOUAISS}, Antônio. \textit{Dicionário eletrônico Houaiss da língua
portuguesa}. Rio de Janeiro: Instituto Antônio Houaiss. Produzido e
distribuído pela Editora Objetiva Ltda.

\tit{MELO}, Gladstone Chaves de. \textit{A língua e o estilo de Rui Barbosa.}
[Col. Rex] Rio de Janeiro:  Simões,  1950.

\tit{MORAIS SILVA}, António de. \textit{Grande dicionário da língua
portuguesa}. 10 ed.~rev.~cor.~muito aum.~e actualizada por Augusto
Moreno, Cardoso Júnior e José Pedro Machado. Lisboa: Confluência, 1954.

\tit{PINTO}, Pedro. \textit{Locuções e expressões na ``Réplica'' de Rui
Barbosa} [Col. Rex] Rio de Janeiro: Simões, 1954.

\tit{Vocabulário Ortográfico da Língua Portuguesa}, 4ª ed. Rio de
Janeiro: Academia Brasileira de Letras.\\
Disponível em: http://www.academia.org.br/abl/

\vspace*{1ex}

\textbf{Obras literárias:}

\tit{ALIGHIERI}, Dante (1998) \textit{A divina comédia}. Edição bilíngue,
tradução e notas de Italo Eugenio Mauro. 14ª reimpressão, São Paulo:
Editora 34,  2007.

\tit{BARBOSA}, Ruy. \textit{Oração aos moços}. Edição promovida pelo Mensario
Academico Dionysos (s.l.) Casa editora ``O livro''. 1921.

\titidem. \textit{Oração aos moços}. Rio de Janeiro: A. dos Reis, 1932.

\titidem. \textit{Oração aos moços}. Edição comemorativa do
centenário de nascimento do grande brasileiro. Fac"-símile do texto
original, datilografado e contendo emendas do próprio punho do Autor.
Reitoria da Universidade de São Paulo: São Paulo, 1949a.

\titidem. \textit{Oração aos moços}. Estabelecimento do texto,
prefácio e breves notas explicativas por Carlos Henrique da Rocha Lima.
Rio de Janeiro: Casa de Rui Barbosa, 1949b.

\titidem. \textit{Oração aos moços}. Estabelecimento do texto e
notas de Adriano da Gama Kury. Prefácio de Edgard Batista Pereira. Rio
de Janeiro: Casa de Rui Barbosa, 1956.

\titidem. \textit{Oração aos moços}. Edição popular anotada por
Adriano da Gama Kury. 5 ed, Rio de Janeiro: Casa de Rui Barbosa, 1999.

\tit{CAMÕES}, Luís de (1572) \textit{Os lusíadas.} Edição fac"-similar com
estudo filológico de Leodegário de Azedo Filho. Rio de Janeiro:
Francisco Alves, 2007.

\tit{GANDRA}, José Ruy. ``De tanto ver triunfar as nulidades\ldots{}''.
\textit{Revista Época}. Nº. 434, 11 de setembro. São Paulo:
Editora Globo, 2006.

\tit{LIMA}, João Gabriel de. “O maior brasileiro da história”. \textit{Revista
Época}. Nº.~434, 11 de setembro. São Paulo: Editora Globo, 2006.

\tit{MELO}, J. Soares de.  \textit{História da oração aos moços}. Rio de
Janeiro: Fundação Casa de Rui Barbosa, 1974.
\end{bibliohedra}




