\textbf{Rui Barbosa} (Salvador, 1849--Petrópolis, 1923), escritor, jornalista, 
jurista, diplomata, tradutor e político brasileiro, foi um dos mais destacados 
e influentes estadistas que o Brasil já teve. Um dos fundadores da Academia 
Brasileira de Letras, liderou em 1907 a delegação brasileira na Conferência 
de Paz, em Haia, Holanda. Lá, sua participação brilhante lhe valeu renome 
internacional e a alcunha de “Águia de Haia”. Em 1893, por combater o golpe 
que levou Floriano Peixoto à presidência, teve de se exilar, primeiro em  
Buenos Aires, depois em Lisboa e, por fim, em Londres, onde permanece até 
1895, e de onde contribuía para a imprensa brasileira com uma série de 
artigos mais tarde publicados sob o título de \textit{Cartas de Inglaterra} 
(1896). Concorreu duas vezes à presidência do Brasil, e em 1921 foi eleito 
juiz da Corte Internacional de Justiça. Sua obra, rica e extensa, abrange 
vários campos do saber, e inclui os \textit{Comentários à Constituição 
Federal Brasileira}, \textit{O Elogio de Castro Alves} (1881), 
\textit{Visita à terra natal} (1893), \textit{Discursos e conferências} 
(1907) e a célebre \textit{Oração aos moços} (1920).

\textbf{Oração aos moços} é um dos mais célebres discursos de Rui
Barbosa, escrito para paraninfar os formandos da
turma de 1920 da Faculdade de Direito do Largo de São Francisco, em São
Paulo. Impedido de comparecer,  por problemas de saúde, o texto foi
lido pelo professor Reinaldo Porchat. Trata-se de uma das mais
brilhantes reflexões produzidas pelo jurista sobre o papel do
magistrado e a missão do advogado. O autor faz um balanço de sua vida
como advogado, jornalista e político, como exemplo para as novas
gerações. Esta edição traz ainda, em apêndice, a famosa carta
a Evaristo de Morais que ficaria conhecida como ``O dever do advogado'',
na qual Rui trata com a propriedade e a elegância que lhe são peculiares
dos dilemas de ética profissional com os quais se deparam os que 
seguem a carreira jurídica. 


\textbf{Marcelo Módolo} é doutor em Filologia e Língua Portuguesa pela
Universidade de São Paulo, onde leciona atualmente. Suas pesquisas
recentes se relacionam à crítica textual de manuscritos modernos,
história do português brasileiro e sintaxe. Coordena a publicação 
das obras inéditas do gramático e linguista Celso Pedro Luft, 
pela Editora Globo, e a Equipe de Linguística de \textit{Corpus} 
do Projeto de História do Português Paulista, o
Projeto Caipira, sediado na \textsc{usp}.

\textbf{Pedro Luso} é advogado civilista, tendo colaborado para \textit{A Tribuna}, 
de  Blumenau, Santa Catarina, e para o \textit{Jornal do Comércio} de Porto Alegre.  
Em 1971, formou-se em  Direito pela Pontifícia Universidade Católica, 
em Porto Alegre.  Em 1972,  passa a exercer o cargo de procurador do 
Instituto de Previdência  Social do Estado do Rio Grande do Sul.  
Concluiu a pós-graduação em Direito na Escola da Magistratura, em 1973. 


